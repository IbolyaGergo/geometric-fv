%! TeX root: ../main.tex
\section{Explicit}%
\label{sec:Explicit}
The formulation in this chapter primarily follows the wave-propagation algorithms
established by LeVeque \cite{1985_LeVeque_CONF}, \cite{1992_LeVeque_BOOK}, \cite{2002_LeVeque_BOOK},
and Van Leer \cite{1977_VanLeer}.

To establish a baseline for the semi-implicit formulation, we first consider the
classical explicit geometric approach. Following the framework detailed in
\cite{1992_LeVeque_BOOK}, \cite{2002_LeVeque_BOOK} we interpret the finite volume update as a measure of
the mass transported across cell interfaces over a time step $\Delta t$ .

The standard explicit finite volume framework relies on the
Reconstruct-Evolve-Average (REA) algorithm... 

We restrict our discussion to the linear advection equation
\begin{equation}
    \label{eq:linadv}
    u_t + a u_x = 0, \quad x \in \mathbb{R}, \; t > 0,
\end{equation}
for a scalar quantity \(u(t,x)\) transported at a constant positive speed \(a > 0\), subject to the initial condition \(u(0,x) = u_0(x)\).

The exact solution to \eqref{eq:linadv} is given by the method of characteristics as
\begin{equation}
    \label{eq:linadv-exact-sol}
    u(t,x) = u_0(x - a t).
\end{equation}

\subsection{Reconstruct-Evolve-Average Algorithm}
Explicit finite volume schemes for solving \eqref{eq:linadv} can be derived
structurally using the Reconstruct-Evolve-Average (REA) algorithm,
a geometric framework detailed extensively in \cite{1992_LeVeque_BOOK,2002_LeVeque_BOOK}.

\paragraph*{The REA Algorithm}
\label{par:REA}
Let \(u_i^n\) denote the cell average over the \(i\)-th cell at time \(t_n\). The update to \(t_{n+1} = t_n + \Delta t\) proceeds in three steps:

\begin{enumerate}
    \item \textbf{Reconstruct:} Construct a global piecewise polynomial function \(\tilde{u}^{n}(x,t_n)\) defined for all \(x\) from the set of discrete cell averages \(\{u_i^n\}\).

    \item \textbf{Evolve:} Solve \eqref{eq:linadv} exactly with \(\tilde{u}^{n}(x,t_n)\) as the initial data to obtain the evolved solution at the next time level:
    \[ \check{u}(x, t_{n+1}) = \tilde{u}^n(x - a\Delta t, t_n). \]

    \item \textbf{Average:} Compute the new cell averages by integrating the evolved function over each grid cell:
    \begin{equation}
        \label{eq:rea-average}
        u_{i}^{n+1} = \frac{1}{\Delta x} \int_{x_{i - 1/2}}^{x_{i + 1/2}} \check{u}(x,t_{n+1}) \,\mathrm{d}x.
    \end{equation}
\end{enumerate}

The order of accuracy and stability of the resulting scheme depend entirely on the choice of the reconstruction \(\tilde{u}^{n}(x,t_n)\).
For instance, the first-order Godunov scheme is obtained via a piecewise constant reconstruction:
\[
    \tilde{u}^{n}(x,t_{n}) = u_{i}^{n} \quad \text{for } x \in [x_{i-1/2}, x_{i+1/2}).
\]

In the first-order case, the solution is reconstructed as a piecewise constant function.
The evolution of the solution is governed by the linear shift of the characteristic curves.
Geometrically, the flux $F_{i+1/2}$ represents the area of the solution profile swept
across the interface $x_{i+1/2}$ during the time interval $[t^n, t^{n+1}]$.
While this explicit formulation yields a stable scheme under the CFL condition $\nu \leq 1$,
our goal in Chapter X is to relax this restriction...

The limitation of explicit methods is the small time step.
