%! TeX root: ../main.tex
\section{Explicit}%
\label{sec:Explicit}
The formulation in this chapter primarily follows the algorithms
established by LeVeque \cite{1985_LeVeque_CONF}, \cite{1992_LeVeque_BOOK}, \cite{2002_LeVeque_BOOK},
and Van Leer \cite{1977_VanLeer}.

We restrict our discussion to the linear advection equation
\begin{equation}
    \label{eq:linadv}
    u_t + a u_x = 0, \quad x \in \mathbb{R}, \; t > 0,
\end{equation}
for a scalar quantity \(u(t,x)\) transported at a constant positive speed \(a > 0\), subject to the initial condition \(u(0,x) = u_0(x)\).

The exact solution to \eqref{eq:linadv} is
\begin{equation}
    \label{eq:linadv-exact-sol}
    u(t,x) = u_0(x - a t).
\end{equation}

\subsection{Reconstruct-Evolve-Average Algorithm}
\label{sub:Reconstruct-Evolve-Average Algorithm}
Explicit finite volume schemes for solving \eqref{eq:linadv} can be derived
using the Reconstruct-Evolve-Average (REA) algorithm, a geometric framework
detailed extensively in \cite{1992_LeVeque_BOOK,2002_LeVeque_BOOK}. The method
was originally proposed by Godunov \cite{1959_Godunov} for solving the nonlinear
Euler equations of gas dynamics.

\paragraph*{The REA Algorithm}
\label{par:REA}
Let \(u_i^n\) denote the cell average over the \(i\)-th cell at time \(t_n\). The update to \(t_{n+1} = t_n + \Delta t\) proceeds in three steps:

\begin{enumerate}
    \item \textbf{Reconstruct:} Construct a global piecewise polynomial function \(\tilde{u}^{n}(x,t_n)\) defined for all \(x\) from the set of discrete cell averages \(\{u_i^n\}\).

    \item \textbf{Evolve:} Solve \eqref{eq:linadv} exactly with \(\tilde{u}^{n}(x,t_n)\) as the initial data to obtain the evolved solution at the next time level:
    \[ \tilde{u}(x, t_{n+1}) = \tilde{u}^n(x - a\Delta t, t_n). \]

    \item \textbf{Average:} Compute the new cell averages by integrating the evolved function over each grid cell:
    \begin{equation}
        \label{eq:rea-average}
        u_{i}^{n+1} = \frac{1}{\Delta x} \int_{x_{i - 1/2}}^{x_{i + 1/2}} \tilde{u}(x,t_{n+1}) \,\mathrm{d}x.
    \end{equation}
\end{enumerate}

\subsection{Example: The First-Order Godunov Scheme}%
\label{sub:Example: The First-Order Godunov Scheme}

To illustrate this procedure, we apply the REA algorithm to the simplest case: a piecewise constant reconstruction.

\textbf{1. Reconstruct:} 
We define the function \(\tilde{u}^n(x, t_n)\) to be constant within each cell, taking the value of the cell average:
\[
    \tilde{u}^{n}(x,t_{n}) = u_{i}^{n} \quad \text{for } x \in [x_{i-1/2}, x_{i+1/2}).
\]
This produces a "staircase" profile representing the solution.

\textbf{2. Evolve:} 
We evolve this profile exactly over one time step \(\Delta t\). Geometrically, the entire staircase shifts to the right by a distance \(a\Delta t\). Assuming the CFL condition \(a\Delta t \leq \Delta x\), the solution at the interface \(x_{i-1/2}\) is determined by the value of the cell to the left, \(u_{i-1}^n\).

\textbf{3. Average:} 
Instead of integrating the function over the whole cell, we can interpret the update via fluxes. The mass in cell \(i\) changes only due to the areas of the profile "swept" across the boundaries.
The total mass swept across the interface \(x_{i-1/2}\) is the width of the shift multiplied by the height of the reconstructed function:
\[
    \text{Mass In} = (a \Delta t) \cdot u_{i-1}^n.
\]
Similarly, the mass leaving the cell at \(x_{i+1/2}\) is:
\[
    \text{Mass Out} = (a \Delta t) \cdot u_{i}^n.
\]
Dividing these mass fluxes by \(\Delta t\) defines the numerical fluxes \(F_{i-1/2} = a u_{i-1}^n\) and \(F_{i+1/2} = a u_{i}^n\). Substituting these into the conservation form yields the standard explicit upwind scheme:
\[
    u_i^{n+1} = u_i^n - \frac{a \Delta t}{\Delta x} (u_i^n - u_{i-1}^n).
\]

The order of accuracy and stability of the resulting scheme depend entirely on the choice of the reconstruction \(\tilde{u}^{n}(x,t_n)\).
For instance, the first-order Godunov scheme is obtained via a piecewise constant reconstruction:
\[
    \tilde{u}^{n}(x,t_{n}) = u_{i}^{n} \quad \text{for } x \in [x_{i-1/2},
    x_{i+1/2}),
\]
where the constant value is the cell average.

In the first-order case, the solution is reconstructed as a piecewise constant function.
The evolution of the solution is governed by the linear shift of the initial
data. Geometrically, the flux $F_{i+1/2}$ represents the area of the initial profile swept
across the interface $x_{i+1/2}$ during the time interval $[t^n, t^{n+1}]$.
In case of the linear advection equation \eqref{eq:linadv} this yields the
numerical flux
\[
    F_{i-1/2} = u_{i-1}^{n},
\]
and the update of the cell average

While this explicit formulation yields a stable scheme under the CFL condition $\nu \leq 1$,
our goal in Chapter X is to relax this restriction...

To obtain a second-order scheme, one might consider a piecewise linear
reconstruction with slopes \(\sigma_{i}\) in each cell. For a third-order
scheme, we choose a piecewise-parabolic reconstruction and so on.
Generally, to obtain an \(n\)-th order scheme we need an \((n-1)\)-th
order polynomial reconstruction.

The limitation of explicit methods is the small time step.
