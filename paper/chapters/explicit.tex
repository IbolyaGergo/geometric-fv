%! TeX root: ../main.tex
\chapter{The Explicit Geometric Framework}%
\label{sec:The Explicit Geometric Framework}
The formulation in this chapter primarily follows the algorithms established by Godunov \cite{godunov1959finite}, Van Leer \cite{van1977towards}, LeVeque \cite{goodman1988geometric, leveque1992numerical, leveque2002finite}.

We restrict our discussion to the linear advection equation
\begin{equation}
    \label{eq:linadv}
    u_t + a u_x = 0, \quad x \in \mathbb{R}, \; t > 0,
\end{equation}
for a scalar quantity \(u(t,x)\) transported at a constant positive speed \(a > 0\), subject to the initial condition \(u(0,x) = u_0(x)\).

The exact solution to \eqref{eq:linadv} is
\begin{equation}
    \label{eq:linadv-exact-sol}
    u(t,x) = u_0(x - a t).
\end{equation}

\section{Reconstruct-Evolve-Average Algorithm}
\label{sub:Reconstruct-Evolve-Average Algorithm}
Explicit finite volume schemes for solving \eqref{eq:linadv} can be derived using the Reconstruct-Evolve-Average (REA) algorithm, a geometric framework detailed extensively in \cite{leveque1992numerical,leveque2002finite}.
The method was originally proposed by Godunov \cite{godunov1959finite} who introduced the concept of evolving a piecewise constant reconstruction for solving the nonlinear Euler equations of gas dynamics.

Van Leer \cite{van1977towards} demonstrated that the algorithm can be extended to higher-order accuracy by replacing Godunov's constant reconstruction with higher-degree polynomials.

\paragraph*{The REA Algorithm}
\label{par:REA}
Let \(u_i^n\) denote the cell average over the \(i\)-th cell at time \(t^n\).
The update to \(t^{n+1} = t^n + \Delta t\) proceeds in three steps:

\begin{enumerate}
    \item \textit{Reconstruct:} Construct a global piecewise polynomial function \(\tilde{u}(x,t^n)\) defined for all \(x\) from the set of discrete cell averages \(\{u_i^n\}\).

    \item \textit{Evolve:} Solve \eqref{eq:linadv} exactly with \(\tilde{u}(x,t^n)\) as the initial data to obtain the evolved solution at the next time level:
    \[
        \tilde{u}(x, t^{n+1}) = \tilde{u}(x - a\Delta t, t^n).
    \]

    \item \textit{Average:} Compute the new cell averages by integrating the evolved function over each grid cell:
    \[
        u_{i}^{n+1} = \frac{1}{\Delta x} \int\limits_{x_{i - 1/2}}^{x_{i + 1/2}} \tilde{u}(x,t^{n+1}) \,\diff x.
    \]
\end{enumerate}

\section{First-Order Godunov Scheme}%
\label{sub:First-Order Godunov Scheme}

To illustrate this procedure, we apply the REA algorithm to the simplest case: a piecewise constant reconstruction.

\begin{enumerate}
    \item \textit{Reconstruct:}
        We define the function \(\tilde{u}(x, t^n)\) to be constant within each cell, taking the value of the cell average (see Figure \ref{fig:godunov_full_page}, top):
    \[
        \tilde{u}(x,t^{n}) = u_{i}^{n} \quad \text{for } x \in [x_{i-1/2}, x_{i+1/2}).
    \]
    This produces a "staircase" profile representing the solution.
    \item \textit{Evolve:}
    We evolve this profile exactly over one time step \(\Delta t\).
    Geometrically, the entire staircase shifts to the right by a distance \(a\Delta t\) (see Figure \ref{fig:godunov_full_page}, middle):
    \[
        \tilde{u}(x, t^{n+1}) = \tilde{u}(x - a\Delta t, t^n).
    \]
\item \textit{Average:}
    We compute the new cell average by directly integrating the evolved profile (see Figure \ref{fig:godunov_full_page}, bottom):
    \begin{equation}
        \begin{split}
            \label{eq:godunov_integral}
            u_{i}^{n+1} &= \frac{1}{\Delta x} \int\limits_{x_{i - 1/2}}^{x_{i + 1/2}} \tilde{u}(x,t^{n+1}) \,\diff x\\
                        &= \frac{1}{\Delta x} \left( \int_{x_{i-1/2}}^{x_{i-1/2}+a\Delta t} u_{i-1}^n \,\diff x + \int_{x_{i-1/2}+a\Delta t}^{x_{i+1/2}} u_i^n \,\diff x \right)\\
                        &= \frac{1}{\Delta x} \left(u_{i-1}^{n} a\Delta t + u_{i}^{n} \left(\Delta x - a\Delta t\right)\right)\\
                        &= u_{i-1}^{n} \nu + u_{i}^{n} \left(1-\nu\right),
        \end{split}
    \end{equation}
\end{enumerate}
where \(\nu = a\Delta t / \Delta x\) is the Courant number.
See Figure~\ref{fig:godunov_full_page}.

\begin{figure}[p] % 'p' forces a dedicated page
    \centering

    % 1. Reconstruction
    \begin{subfigure}[b]{\textwidth}
        \centering
        \includestandalone[mode=buildnew]{fig/godunov-reconstruct}
        %\caption{} % Captions might be redundant if text is inside the image
    \end{subfigure}

    \vspace{1cm} % Nice breathing room

    % 2. Evolution
    \begin{subfigure}[b]{\textwidth}
        \centering
        \includestandalone[mode=buildnew]{fig/godunov-evolve}
    \end{subfigure}

    \vspace{1cm}

    % 3. Averaging
    \begin{subfigure}[b]{\textwidth}
        \centering
        \includestandalone[mode=buildnew]{fig/godunov-average}
    \end{subfigure}

    \caption{Solving \(u_t + au_x = 0\) by the Godunov Scheme. (Top) Data is reconstructed as piecewise constant.
    (Middle) The profile shifts by $a\Delta t$.
    (Bottom) The new cell average is computed.}
    \label{fig:godunov_full_page}
\end{figure}

\newpage
\subsection{Numerical flux}

Obtaining the new cell average by integrating the shifted profile is mathematically equivalent to a flux-based update.
The change in the total conserved quantity within cell \(i\) corresponds precisely to the geometric area of the solution profile "swept" across the cell boundaries during the time step \(\Delta t\).

The total quantity entering the cell across the left interface \(x_{i-1/2}\) is the width of the shift multiplied by the height of the reconstructed function in the upwind cell:
\[
    \text{In} = (a \Delta t) \cdot u_{i-1}^n.
\]
Similarly, the quantity leaving the cell across the right interface \(x_{i+1/2}\) is:
\[
    \text{Out} = (a \Delta t) \cdot u_{i}^n.
\]
Dividing these values by \(\Delta t\) yields the numerical fluxes \(F_{i-1/2} = a u_{i-1}^n\) and \(F_{i+1/2} = a u_{i}^n\).
Substituting these into the standard conservation form \eqref{eq:fvm_general} recovers the classical explicit upwind scheme:
\begin{equation}
    \label{eq:godunov_flux}
    u_i^{n+1} = u_i^n - \nu (u_i^n - u_{i-1}^n).
\end{equation}
Equation \eqref{eq:godunov_flux} is equivalent to \eqref{eq:godunov_integral}.

\begin{figure}[ht]
    \centering
    \includestandalone[mode=buildnew]{fig/godunov-flux}
    \caption{Flux-based interpretation of the Godunov scheme.}
    \label{fig:godunov_flux}
\end{figure}


The order of accuracy and stability of the resulting scheme depend entirely on the choice of the reconstruction \(\tilde{u}(x,t^n)\).
For instance, the first-order Godunov scheme is obtained via a piecewise constant reconstruction.
To obtain a second-order scheme, one might consider a piecewise linear reconstruction with slopes \(\sigma_{i}\) in each cell.
For a third-order scheme, we choose a piecewise-parabolic reconstruction and so on.
Generally, to obtain an \(n\)-th order scheme we need an \((n-1)\)-th order polynomial reconstruction.

\section{Second-Order Schemes}
\label{sub:second_order}
To achieve high-order accuracy, we exploit the generality of Step 1 in the REA algorithm.
Instead of a piecewise constant reconstruction, we specify a piecewise linear function.

\textit{1. Reconstruct:}
In each cell \(\Omega_i\), we define a linear profile with slope \(\sigma_i\):
\[
    \tilde{u}(x, t^n) = u_i^n + \sigma_i (x - x_i), \quad \text{for } x \in [x_{i-1/2}, x_{i+1/2}),
\]
where \(x_i\) is the cell center.
The resulting scheme is determined entirely by the choice of the slope \(\sigma_i\).

\textit{2. Evolve:}
As in the first-order case, the linear advection equation dictates that this profile shifts rigidly by \(a \Delta t\).

\textit{3. Average (Geometric Flux Calculation):}
The flux calculation remains the determination of the quantity swept across the interface.
However, due to the non-zero slope, the geometric shape swept across the boundary is no longer a rectangle, but a \textit{trapezoid}.

Consider the interface \(x_{i+1/2}\).
The quantity leaving cell \(i\) is the area under the linear reconstruction over the interval \([x_{i+1/2} - a\Delta t, x_{i+1/2}]\).
The base of this trapezoid is the distance traveled, \(a \Delta t\).
The average height of the trapezoid is the value of the reconstructed function at the midpoint of this interval:
\[
    x_{mid} = x_{i+1/2} - \frac{1}{2} a \Delta t.
\]
Evaluating the reconstruction \(\tilde{u}\) at this midpoint yields the time-averaged value at the interface:
\[
    u_{i+1/2}^* = u_i^n + \sigma_i \left( \frac{\Delta x}{2} - \frac{a \Delta t}{2} \right).
\]
The resulting numerical flux is simply the transport speed multiplied by this value:
\begin{equation}
    \label{eq:linear_flux}
    F_{i+1/2} = a u_i^n + \frac{1}{2} a (1 - \nu) \Delta x \sigma_i.
\end{equation}

\begin{figure}[ht]
    \centering
    \includestandalone[mode=buildnew]{fig/linear-flux}
    \caption{Geometric interpretation of the second-order flux.
    The shaded trapezoids represent the total quantity transported across the interfaces during \(\Delta t\).
    Unlike the Godunov scheme, the varying height of the reconstruction introduces a second-order correction to the swept area.}
    \label{fig:linear_flux}
\end{figure}

Different standard schemes correspond to specific choices of the slope \(\sigma_i\):
\begin{itemize}
    \item \textit{Beam-Warming:} Choosing the upwind (backward) slope \(\sigma_i = (u_i^n - u_{i-1}^n)/\Delta x\).
    \item \textit{Lax-Wendroff:} Choosing the downwind (forward) slope \(\sigma_i = (u_{i+1}^n - u_i^n)/\Delta x\).
    \item \textit{Fromm:} Choosing the centered slope \(\sigma_i = (u_{i+1}^n - u_{i-1}^n)/ \left( 2\Delta x \right)\).
\end{itemize}

\begin{figure}[h]
    \centering
    \includestandalone[mode=buildnew]{fig/linear-slopes}
    \caption{}
    \label{fig:linear_slopes}
\end{figure}

However, all these choices produce unphysical oscillations in the vicinity of large gradients.
All high-order, linear (constant coefficient) schemes share this major limitation.
As established by Godunov's theorem, a linear, monotonicity-preserving scheme can be at most first-order accurate \cite{godunov1959finite}.

To resolve this conflict between accuracy and stability, we must abandon linear reconstructions.
But before constructing such a mechanism, we must define the stability property we wish to enforce.

\section{Monotonicity and the Upwind Range Condition}
\label{sec:monotonicity_toro}
The oscillations observed in high-order linear schemes represent a violation of the physics.
To construct a robust method, we must impose a stability criterion.

\subsection{Monotonicity Preserving Schemes}
A numerical scheme is said to be \textit{Monotonicity Preserving (MP)} if it maintains the monotonic nature of the initial data.
Formally, if the data at time \(t^n\) is monotonic (i.e., \(u_i^n \geq u_{i+1}^n\) for all \(i\), or \(u_i^n \leq u_{i+1}^n\) for all \(i\)), then the updated solution at \(t^{n+1}\) must also be monotonic.

While this global definition is fundamental, it is difficult to enforce directly during the construction of a scheme.
Furthermore, monotonicity is a global condition.
Since we know that the exact solution of the advection equation \eqref{eq:linadv} is influenced only by the values in its domain of dependence, we seek a local sufficient condition that guarantees this global property.

\subsection{The Upwind Range Condition}

For the linear advection equation with constant speed \(a > 0\), the physical solution propagates along the characteristics from the upwind direction (from \(i-1\) to \(i\)).
The updated average \(u_i^{n+1}\) in the \(i\)-th cell should be bounded by the current value \(u_i^n\) and the upwind neighbor \(u_{i-1}^n\).
Under the CFL condition \( \nu \leq 1 \), this implies:
\begin{equation}
    \min(u_{i-1}^n, u_i^n) \leq u_i^{n+1} \leq \max(u_{i-1}^n, u_i^n).
\end{equation}

The condition was formalized in many different ways in the literature: upwind range condition\cite{1998_Laney_BOOK},
data compatibility condition \cite{2009_Toro_BOOK}, upwind monotonic property \cite{1989_Huynh_CONF}, local boundedness property \cite{1982_Roe,1997_Thuburn}.

This condition can be expressed as:
\begin{equation}
    \label{eq:upwind_range_condition_explicit}
    0 \leq \frac{u_i^{n+1} - u_i^n}{u_{i-1}^n - u_i^n} \leq 1.
\end{equation}
For a proof see \cite{toro2013riemann}.

We can interpret \eqref{eq:upwind_range_condition_explicit} by considering the maximum permissible change in the solution.
Assume, for example, that the data is increasing locally (\(u_{i-1}^{n} < u_{i}^{n}\)).
Physics dictates that under the CFL condition \(\nu \leq 1\), the new value \(u_{i}^{n+1}\) cannot drop below the old value from the upwind cell \(u_{i-1}^{n}\).
Conversely, the update is bounded from above by \(u_{i}^{n}\), as the solution cannot physically be larger than its current state if the upwind value is larger. Thus, we can write
\[
    u_{i-1}^{n} \leq u_{i}^{n+1} \leq u_{i}^{n}
    \implies
    u_{i-1}^{n}-u_{i}^{n} \leq u_{i}^{n+1} - u_{i}^{n} \leq 0.
\]
Dividing by \(u_{i-1}^{n}-u_{i}^{n} < 0\) yields \eqref{eq:upwind_range_condition_explicit}.

Crucially, this interpretation extends naturally to higher dimensions, where the denominator represents the sum of maximum contributions from all upwind neighbors.

\paragraph*{Example: Godunov Scheme}
Substituting the Godunov scheme \eqref{eq:godunov_flux} into \eqref{eq:upwind_range_condition_explicit}, we obtain:
\[
    0 \leq
    \frac{-\nu (u_i^n - u_{i-1}^n)}{u_{i-1}^n - u_i^n}
    = \frac{\nu (u_{i-1}^n - u_i^n)}{u_{i-1}^n - u_i^n}
    = \nu
    \leq 1
\]
Thus, the condition becomes \(0 \leq \nu \leq 1\), which is exactly the CFL stability condition.
The Godunov scheme satisfies the Upwind Range Condition whenever it is stable.

\paragraph*{Example: Lax-Wendroff Scheme}
The Lax-Wendroff scheme can be written as the Godunov update plus a second-order correction term:
\[
    u_i^{n+1} = u_i^n - \nu (u_i^n - u_{i-1}^n) - \frac{1}{2}\nu(1-\nu) \left[ (u_{i+1}^n - u_i^n) - (u_i^n - u_{i-1}^n) \right].
\]
To verify the condition, we define the local gradient ratio \(r_i\), which measures the smoothness of the data:
\begin{equation}
    \label{eq:smoothness_ratio_explicit}
    r_i = \frac{u_i^n - u_{i-1}^n}{u_{i+1}^n - u_i^n}.
\end{equation}
Substituting the update formula into the Upwind Range Condition \eqref{eq:upwind_range_condition_explicit}, and assuming \(u_{i}^{n}-u_{i-1}^{n} \neq 0\), the ratio becomes:
\[
    0 \leq
    \frac{u_i^{n+1} - u_i^n}{u_{i-1}^n - u_i^n}
    = \nu + \frac{1}{2}\nu(1-\nu) \left(\frac{1}{r_i}- 1\right)
    \leq 1.
\]
Rearranging we get
\[
    -\frac{1+\nu}{1-\nu} \leq \frac{1}{r_i} \leq \frac{\nu+2}{\nu}.
\]
Unlike the Godunov case, for a fixed \(\nu\) the data in general does not satisfy the Upwind Range Condition.
Near a strong discontinuity, \(r_i\) can be very large (steepening gradient) or negative (local extremum).
If \(r_i\) is sufficiently large, the total ratio will exceed 1, violating the upper bound of the Upwind Range Condition.
This mathematically confirms why the linear Lax-Wendroff scheme produces oscillations.

\subsection{Connecting Limiters to the Range Condition}
By limiting the slope such that the resulting geometric fluxes satisfy the Upwind Range Condition, we derive the class of \textit{TVD Limiters} (e.g.,
minmod, superbee, Van Leer).

\paragraph{Implications for Implicit Schemes:}
In Chapter \ref{sec:The Implicit Geometric Framework} we will derive an analogous ratio for the implicit formulation and enforce strict monotonicity without the time-step restrictions of explicit methods.

\section{High-Resolution Schemes}
\label{sec:construction_limiters}
The unphysical oscillations produced by linear high-order schemes can be eliminated by introducing a non-linear mechanism known as \textit{slope limiting}. The fundamental idea is to retain the piecewise-linear reconstruction from Section \ref{sub:second_order}, but to adaptively modify the slope in each cell based on the local smoothness of the solution.

\subsection{The Limited Slope}
In each cell, we assume a linear reconstruction as
\[
    \tilde{u}(x, t^n) = u_i^n + \sigma_i (x - x_i).
\]
Instead of using a fixed formula for the slope \(\sigma_i\), we define it as the product of a high-order reference slope and a limiter function \(\Psi_i\):
\begin{equation}
    \sigma_i = \Psi_i \, \sigma_i^{\text{HO}}.
\end{equation}
The limiter \(\Psi_i\) is a function of the local solution profile, designed to be near 1 in smooth regions (to retain second-order accuracy) and near 0 near discontinuities or extrema (to ensure monotonicity).

\subsection{Deriving Limiter Constraints from the Upwind Range Condition}
The choice of the limiter \(\Psi_i\) is not arbitrary.
It must be constrained to ensure the resulting scheme is monotonicity-preserving.
We derive these constraints by substituting the limited slope into the finite volume update and enforcing the Upwind Range Condition (Equation \eqref{eq:upwind_range_condition_explicit}).

Our choice for the high-order reference slope is the downwind slope
\[
    \sigma_i^{\text{HO}} = \frac{u_{i+1}^n - u_i^n}{\Delta x},
\]
which leads to the Lax-Wendroff scheme when used without a limiter.

The flux for a piecewise-linear reconstruction is given by Equation \eqref{eq:linear_flux}.
Substituting our limited slope yields the limited flux at interface \(x_{i+1/2}\):
\[
    F_{i+1/2} = a u_i^n + \frac{1}{2} a (1 - \nu) \Delta x \, \sigma_i = a u_i^n + \frac{1}{2} a (1 - \nu) (u_{i+1}^n - u_i^n) \Psi_i.
\]
Substituting to the finite volume update formula yields
\begin{align*}
    u_i^{n+1} &= u_i^n - \frac{\Delta t}{\Delta x} (F_{i+1/2} - F_{i-1/2}) \\
    &= u_i^n - \nu(u_i^n - u_{i-1}^n) - \frac{\nu(1-\nu)}{2} \left[ \Psi_i(u_{i+1}^n - u_i^n) - \Psi_{i-1}(u_i^n - u_{i-1}^n) \right].
\end{align*}
Assuming \(u_{i-1}^{n} - u_{i}^{n} \neq 0\), after some algebraic manipulation we reproduce the ratio from \eqref{eq:upwind_range_condition_explicit}
\[
    \frac{u_i^{n+1} - u_i^n}{u_{i-1}^n - u_i^n} =
    \nu + \frac{\nu(1-\nu)}{2} \left( \frac{\Psi_i}{r_i} - \Psi_{i-1} \right),
\]
where
\[
    r_i = \frac{u_{i}^{n}-u_{i-1}^{n}}{u_{i+1}^{n}-u_{i}^{n}}.
\]
The Upwind Range Condition \eqref{eq:upwind_range_condition_explicit} demands that this entire expression must lie in the interval \([0, 1]\)
\[
    0 \leq
    \frac{u_i^{n+1} - u_i^n}{u_{i-1}^n - u_i^n} =
    \nu + \frac{\nu(1-\nu)}{2} \left( \frac{\Psi_i}{r_i} - \Psi_{i-1} \right)
    \leq 1.
\]
After rearranging we get
\[
    -\frac{2}{1-\nu} \le \frac{\Psi_i}{r_i} - \Psi_{i-1} \le \frac{2}{\nu}.
\]
% TODO: fig for the downwind flux restriction
We can further constrain the limiter so that the numerical flux is bounded by the Godunov upwind flux \(a u_{i}^{n}\) and the downwind flux \(a u_{i+1}^{n}\)
\begin{align*}
    a u_{i}^{n} &\leq  F_{i+1/2} \leq a u_{i+1}^{n} \\
    a u_{i}^{n} &\leq a u_i^n + \frac{1}{2} a (1 - \nu) (u_{i+1}^n - u_i^n) \Psi_i \leq a u_{i+1}^{n}
\end{align*}
which, provided \(u_{i+1}^{n} - u_{i}^{n} \neq 0\) yields
\[
    0 \le \Psi_i \le \frac{2}{1-\nu}.
\]
Since this must hold also for \(\Psi_{i-1}\), assuming the worst case scenario, we can derive the sufficient condition
\[
    0 \le \frac{\Psi_i}{r_i} \le \frac{2}{\nu}.
\]
Collecting these results, we have derived a set of CFL-dependent sufficient conditions for \(r>0\):
\[
    0 \le \Psi_i \le \frac{2}{1-\nu} \quad \text{and} \quad \Psi_i \le \frac{2r_i}{\nu}.
\]
To create a single, universally applicable limiter, we take the most restrictive form of these bounds across all valid Courant numbers (\(0 \le \nu \le 1\)).
\begin{itemize}
    \item The bound \(\Psi_i \le \frac{2}{1-\nu}\) is most restrictive as \(\nu \to 0^+\), where it becomes \(\Psi_i \le 2\).
    \item The bound \(\Psi_i \le \frac{2r}{\nu}\) is most restrictive as \(\nu \to 1^-\), where it becomes \(\Psi_i \le 2r\).
\end{itemize}
By combining these most restrictive, CFL-independent bounds, we arrive at the final set of sufficient conditions for the limiter function \(\Psi_i\) that guarantees monotonicity for \(r>0\):
\begin{equation}
\label{eq:sweby_condition}
    0 \le \Psi_i \le 2 \quad \text{and} \quad \Psi_i \le 2r_i.
\end{equation}
Furthermore, for non-monotonic data (\(r \le 0\)), we require
\[
    \Psi_i = 0.
\]
% TODO: fig TVD region
These inequalities together define the famous TVD region, ensuring the scheme is both second-order accurate in smooth regions and robustly monotonic across discontinuities.

Common examples of limiters satisfying these conditions include:
\begin{itemize}
    \item \textit{MinMod:} \(\Psi(r) = \max(0, \min(1, r))\)
    \item \textit{Superbee:} \(\Psi(r) = \max(0, \min(1, 2r), \min(2, r))\)
    \item \textit{Van Leer:} \(\Psi(r) = \frac{r + |r|}{1 + |r|}\)
\end{itemize}

