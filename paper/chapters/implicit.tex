%near start,! TeX root: ../main.tex
\chapter{The Implicit Geometric Framework}%
\label{sec:The Implicit Geometric Framework}

The explicit finite volume schemes discussed in the previous chapter are limited by the Courant-Friedrichs-Lewy (CFL) condition, requiring \(0 \leq \nu \leq 1\).
For many practical applications~--~such as flows with varying velocities or grids with variable sizes~--~this restriction forces the use of impractically small time steps.
To overcome this barrier, we turn to \textit{implicit} methods.

In this chapter, we extend the geometric arguments established for explicit methods to construct high-order implicit schemes.
We demonstrate that the monotonicity constraints derived in the explicit framework have direct analogues in the implicit setting.
A key contribution of this work is the formulation of an \textit{Implicit Upwind Range Condition (IURC)}, a geometric criterion analogous to the well-known condition for explicit schemes.
We show that this condition provides an intuitive pathway for deriving novel, probably monotonic implicit limiters
We begin by analyzing the classical first-order implicit upwind scheme to establish the baseline for stability and accuracy, before deriving our high-resolution implicit extensions.

\section{The First-Order Implicit Upwind Scheme}
\label{sec:implicit_upwind}
Let us consider the scheme
\begin{equation}
    \label{eq:implicit_upwind_cons}
    u_{i}^{n+1} = u_{i}^{n} - \nu\left(u_{i}^{n+1} - u_{i-1}^{n+1}\right).
\end{equation}
This is often referred to as the \textit{implicit upwind} or \textit{backward-time backward-space (BTBS)} scheme.
Unlike explicit methods, the values from the new time step \(t^{n+1}\) appear on both sides of the equation, coupling the grid points into a linear system.
It has the remarkable property, that it is stable for \(\nu \geq 0\).

% TODO: Plot explicit upwind vs implicit upwind to show the difference

By rearranging \eqref{eq:implicit_upwind_cons} we solve for the new average \(u_{i}^{n+1}\):
\begin{equation}
    \label{eq:implicit_upwind_solution}
    u_{i}^{n+1} = \frac{1}{1+\nu}u_{i}^{n} + \frac{\nu}{1+\nu}u_{i-1}^{n+1}.
\end{equation}
If we supply inflow boundary conditions and sweep from the upwind direction, \(u_{i-1}^{n+1}\) is already known when computing \(u_{i}^{n+1}\).
Thus, the only unknown in equation \eqref{eq:implicit_upwind_solution} is \(u_{i}^{n+1}\).

We observe that the solution \(u_{i}^{n+1}\) is a linear interpolation between \(u_{i}^{n}\) and \(u_{i-1}^{n+1}\).
In contrast, the explicit scheme is a linear interpolation between the old values \(u_{i}^{n}\) and \(u_{i-1}^{n}\) (see Equation \eqref{eq:godunov_integral}).

\subsection{Von Neumann Stability Analysis}
To formally verify the stability of the scheme, we perform a von Neumann stability analysis. 
For a comprehensive treatment of the theory, see, e.g., \cite{hirsch2007numerical, leveque2002finite}.
We consider the evolution of a single Fourier mode. 
To avoid confusion with the imaginary unit \( \mathrm{i} = \sqrt{-1} \), we use the index \( j \) for spatial position in this subsection:
\[
    u_j^n = \hat{u}^n e^{\mathrm{i} k j \Delta x},
\]
where \(k\) is the wave number.
The solution at the next time step is related by an amplification factor \(G\), such that \(u_j^{n+1} = G \hat{u}^n e^{\mathrm{i} k j \Delta x}\). 
Substituting this into the scheme \eqref{eq:implicit_upwind_cons} yields:
\begin{align*}
    G \hat{u}^n e^{\mathrm{i} k j \Delta x} &= \hat{u}^n e^{\mathrm{i} k j \Delta x} - \nu \left( G \hat{u}^n e^{\mathrm{i} k j \Delta x} - G \hat{u}^n e^{\mathrm{i} k (j-1) \Delta x} \right).
\end{align*}
Dividing by \(\hat{u}^n e^{\mathrm{i} k j \Delta x}\) and rearranging gives:
\begin{align*}
    G &= 1 - \nu G (1 - e^{-\mathrm{i} k \Delta x}) \\
    G (1 + \nu(1 - e^{-\mathrm{i} k \Delta x})) &= 1.
\end{align*}
Solving for the amplification factor \(G\) results in:
\begin{equation}
    \label{eq:implicit_upwind_g}
    G = \frac{1}{1 + \nu(1 - e^{-\mathrm{i}\theta})},
\end{equation}
where \(\theta = k \Delta x\) is the dimensionless wave number. 
To analyze stability, we examine the magnitude of \(G\). The squared magnitude of the denominator is:
\begin{align*}
    |1 + \nu(1 - \cos\theta + \mathrm{i}\sin\theta)|^2 &= |(1 + \nu(1-\cos\theta)) + \mathrm{i}\nu\sin\theta|^2 \\
    &= (1 + \nu(1-\cos\theta))^2 + (\nu\sin\theta)^2 \\
    &= 1 + 2\nu(1-\cos\theta) + \nu^2(1-\cos\theta)^2 + \nu^2\sin^2\theta \\
    &= 1 + 2\nu(1-\cos\theta) + \nu^2(1 - 2\cos\theta + \cos^2\theta + \sin^2\theta) \\
    &= 1 + 2\nu(1-\cos\theta) + 2\nu^2(1 - \cos\theta) \\
    &= 1 + 2(\nu + \nu^2)(1-\cos\theta).
\end{align*}
For any Courant number \(\nu \geq 0\), the term \(2(\nu + \nu^2)\) is non-negative. 
The term \(1-\cos\theta\) is also always non-negative. 
Therefore, the denominator's squared magnitude is always greater than or equal to 1. 
Since \(|G|^2 = 1 / |\text{denominator}|^2\), we have:
\[
    |G|^2 \leq 1 \quad \text{for all } \nu \geq 0 \text{ and for all } \theta.
\]
This proves that the first-order implicit upwind scheme is \textit{unconditionally stable}.

The magnitude \(|G|\) also reveals the scheme's dissipative nature. 
The scheme is most dissipative for the highest-frequency waves representable on the grid, where \(\theta = \pi\). 
At this frequency, \( \cos \theta = -1 \), and the denominator squared becomes \( 1 + 4\nu + 4\nu^2 = (1+2\nu)^2 \). 
Consequently, \(|G| = 1/(1+2\nu)\). 
As \(\nu\) increases, this value decreases, meaning the scheme strongly damps short-wavelength oscillations. 
While this ensures robustness, it is the cause of the significant numerical diffusion observed in solutions.

While the magnitude \(|G|\) determines how the scheme modifies the \emph{amplitude} of a wave (dissipation), the complex argument \(\arg(G)\) determines how it modifies the \emph{propagation speed} (dispersion).

Ideally, a physical wave with wave number \(k\) traveling at velocity \(a\) shifts its phase by \(-ak\Delta t = -\nu\theta\) over one time step.
The numerical scheme, however, shifts the phase by an angle \(\phi = \arg(G)\).
The ratio of the numerical phase speed \(c_{num}\) to the physical speed \(a\) is defined as:
\begin{equation}
    \frac{c_{num}}{a} = \frac{\phi}{-\nu\theta}.
\end{equation}
If this ratio is \(1\), the error is zero. If \(<1\), the numerical wave moves too slowly (phase lag); if \(>1\), it moves too fast (phase lead).

Recall the amplification factor derived in Equation \eqref{eq:implicit_upwind_g}:
\[
    G = \frac{1}{1 + \nu(1 - \cos\theta) + \mathrm{i}\nu\sin\theta}.
\]
To find the argument \(\phi\), we use the property that for a complex number \(z = 1/D\), the angle is \(\arg(z) = -\arg(D)\).
Let \(D = R + \mathrm{i}I\), where the real part \(R = 1 + \nu(1-\cos\theta)\) and the imaginary part \(I = \nu\sin\theta\).
The numerical phase shift is:
\begin{equation}
    \phi = -\arctan\left( \frac{I}{R} \right) = -\arctan\left( \frac{\nu\sin\theta}{1 + \nu(1 - \cos\theta)} \right).
\end{equation}

By analyzing the ratio \(c_{num}/a = -\phi / (\nu\theta)\), we can determine the dispersion characteristics:

\begin{enumerate}
    \item \textit{Long Wave Limit (\(\theta \to 0\)):}
    For well-resolved waves, \(\sin\theta \approx \theta\) and \(\cos\theta \approx 1\).
    The argument of the arctangent becomes \(\approx \nu\theta / 1\).
    Thus, \(\phi \approx -\nu\theta\), and the relative phase speed approaches \(1\).
    This confirms the scheme is consistent: long waves travel at the correct speed.

    \item \textit{Short Wave Limit (\(\theta \to \pi\)):}
    For the highest frequencies, the denominator \(R\) becomes large (\(1 + 2\nu\)).
    This reduces the effective phase angle compared to the ideal linear growth.
    Calculations show that for all \(\nu > 0\), the magnitude of the numerical phase shift is smaller than the physical shift: \(|\phi| < |-\nu\theta|\).
\end{enumerate}

\textit{Physical Interpretation:}
Since \(|\phi| < \nu\theta\), the relative phase speed is strictly less than 1 for all wave numbers.
This indicates that the implicit upwind scheme introduces a \textit{phase lag}.
Numerical waves travel slower than the physical flow.
While the scheme is heavily dissipative (which tends to kill off these errors before they accumulate), this phase lag is a characteristic feature of implicit schemes.
Combined with the dissipation analysis, we conclude that the first-order implicit upwind scheme tends to both smear out features (diffusion) and retard their propagation speed (dispersion).

\subsection{Geometric and Physical Interpretation}%
\label{sub:implicit_upwind_geom_interp}

To gain more insight, we visualize the old values along with the new values at the old time level \(t^n\).
Since the solution travels along characteristics, we can trace back from the new time level to the old one.
The boundaries from which the new average \(u_{i}^{n+1}\) is computed correspond to the boundaries of the grid cell \(i\) shifted backwards by \(a\Delta t\).

\begin{figure}[htbp]
    \centering

    \begin{subfigure}[b]{\textwidth}
        \centering
        \includestandalone[mode=buildnew]{fig/backtrace}
        %\caption{}
    \end{subfigure}

    \vspace{1cm}
    
    \begin{subfigure}[b]{\textwidth}
        \centering
        \includestandalone[mode=buildnew]{fig/backtrace-front}
        %\caption{}
    \end{subfigure}

    \caption{Visualization of the backward characteristic tracing.
        (Top) Characteristics are traced backward from \(t^{n+1}\) to \(t^n\).
        (Bottom) The implicit reconstruction using values from the new time level.
        Note that the value \(u_i^{n+1}\) originates from the upstream location shifted by \(a\Delta t\).}
    \label{fig:backtrace}
\end{figure}

As illustrated in Figure \ref{fig:backtrace}, this geometric view explains the opposing accuracy behavior.
In the implicit case, as the time step (and thus \(\nu\)) increases, the value \(u_{i-1}^{n+1}\) becomes a less accurate approximation of the inflow than \(u_{i-1}^{n}\).
In contrast, for the explicit upwind scheme, as \(\nu \to 1\), the inflow is better approximated by the upwind neighbor \(u_{i-1}^{n}\).

\subsection{Numerical Flux and Reconstruction Inconsistency}%
\label{sub:implicit_upwind_flux}

Unlike the explicit scheme, the implicit upwind scheme cannot be strictly interpreted as evolving a reconstruction of the data at \(t^n\).
Instead, it is best understood through the geometric definition of the \textit{numerical flux}.

Recall that the finite volume update requires the time-averaged flux across the interface:
\[
    F_{i+1/2} \approx \frac{1}{\Delta t} \int_{t_n}^{t_{n+1}} f(u(x_{i+1/2}, t)) \, \diff t.
\]
The implicit case differs in the approximation of this integral.
Specifically, to estimate the fluxes, we assume a piecewise constant profile determined by the \textit{new} averages:
\[
    \tilde{u}(x,t^n) = u_{i}^{n+1}.
\]
For positive velocity \(a > 0\), the flow across the interface \(x_{i+1/2}\) is determined by the upwind cell \(i\). Hence, the numerical flux is:
\begin{equation}
    \label{eq:implicit-upwind-flux}
    F_{i+1/2} = a u_{i}^{n+1}.
\end{equation}

\begin{figure}[ht]
    \centering

    \begin{subfigure}[b]{\textwidth}
        \centering
        \includestandalone[mode=buildnew]{fig/implicit-upwind-reconstruct}
    \end{subfigure}

    \vspace{1cm}
    
    \begin{subfigure}[b]{\textwidth}
        \centering
        \includestandalone[mode=buildnew]{fig/implicit-upwind-flux}
    \end{subfigure}

    \caption{(Top) The piecewise-constant data profile \(\tilde{u}(x,t^n)\) used to define the numerical flux in the implicit upwind scheme \eqref{eq:implicit_upwind_cons}.
    (Bottom) The numerical flux is the area swept (red) by this reconstruction across the interfaces during the evolution.}
    \label{fig:implicit_upwind_flux}
\end{figure}

This reveals a critical property of the implicit upwind scheme: it employs a reconstruction that is inconsistent with the data at time \(t^n\).
Specifically, the cell average of the reconstruction is not the cell average of the data at time \(t^n\):
\[
    \frac{1}{\Delta x}\int\limits_{{x_{i-1/2}}}^{{x_{i+1/2}}} {\tilde{u}(x,t^n)} \,\diff x = u_{i}^{n+1} \neq u_{i}^{n}.
\]

This inconsistency is a primary source of the scheme's low accuracy and high numerical diffusion.
In case of the implicit upwind scheme, the reconstruction serves solely to define the \textit{fluxes}, not to represent the evolution of the specific fluid elements currently in the cell.

To overcome these limitations, we must develop higher-order schemes based on more sophisticated, consistent reconstructions, which we explore in the next section.

\section{Second-Order implicit schemes}%
\label{sec:second_order_implicit}

% TODO: Plot result second order vs implicit upwind.
%"Caption: "Comparison of numerical solutions for the advection of a Gaussian pulse.
% The first-order scheme exhibits significant numerical diffusion, while the second-order scheme remains much closer to the exact solution.
To construct a second-order scheme, we replace the piecewise-constant profile with a piecewise-linear one.
A key challenge is defining an appropriate slope, \(\sigma_i\), for each cell.
To construct an implicit scheme, we can use the yet-unknown value \(u_i^{n+1}\) to define the slope.
We define the slope such that the linear profile at time \(t^n\), when evaluated at \(x_i\), is equal to the old value \(u_{i}^{n}\), and when it is evaluated at the back-traced position \(x_i - a\Delta t\), is equal to the new cell average \(u_i^{n+1}\).
This yields a reconstruction:
\[
    \tilde{u}(x, t^n) = u_i^n + \sigma_i (x - x_i).
\]
For this to hold, we require \(\tilde{u}(x_i - a\Delta t, t^n) = u_i^{n+1}\), which gives:
\[
    u_i^n + \sigma_i (x_i - a\Delta t - x_i) = u_i^{n+1} \implies \sigma_i = \frac{ u_{i}^{n}-u_{i}^{n+1} }{a\Delta t}.
\]
This choice of slope provides a direct physical link between the spatial variation within the cell and the temporal change of its average value.
Crucially, since the reconstruction has a value of \(u_i^n\) at the cell's midpoint, it satisfies the integral average constraint, resolving the inconsistency of the first-order scheme.
This reconstruction is illustrated in Figure~\ref{fig:box_reconstruct}. 
\begin{figure}[htbp]
    \centering

    \begin{subfigure}[b]{\textwidth}
        \centering
        \includestandalone[mode=buildnew]{fig/box-reconstruct}
    \end{subfigure}

    \vspace{1cm}
    
    \begin{subfigure}[b]{\textwidth}
        \centering
        \includestandalone[mode=buildnew]{fig/box-flux}
    \end{subfigure}

    \caption{(Top) Second-order implicit, piecewise-linear reconstruction at time \(t^n\).
        The slope \(\sigma_i\) is chosen such that the line passes through the old average \(u_{i}^{n}\) (black dot) at the cell center and the new average \(u_i^{n+1}\) (green dot) at the back-traced location \(x_i - a\Delta t\).
(Bottom) Numerical flux for the second-order implicit scheme.
The red trapezoids represent the area swept across the cell interfaces by the linear reconstruction over the time step.}
    \label{fig:box_reconstruct}
\end{figure}

The linear reconstruction (see Section \ref{sub:second_order}) yields a flux
\begin{equation}
    \begin{split}
        \label{eq:box-flux-explicit-start}
        F_{i+1/2} &= a u_i^n + \frac{1}{2} a (1 - \nu) \Delta x \sigma_i \\
                  &= a u_{i}^{n} + \frac{1}{2} a (1-\nu) \Delta x \frac{ u_{i}^{n}-u_{i}^{n+1} }{a\Delta t} \\
                  &= a u_{i}^{n} + \frac{1}{2} a \frac{1-\nu}{\nu}  \left( u_{i}^{n}-u_{i}^{n+1} \right)
    \end{split}
\end{equation}
A corresponding expression can be written for the flux \(F_{i-1/2}\) at the left-hand interface by shifting the index from \(i\) to \(i-1\).


Substituting these fluxes into the general finite volume update formula, Equation \eqref{eq:fvm_general}, gives the update formula for the new scheme:
\begin{equation*}
    \begin{split}
        u_{i}^{n+1} &= u_{i}^{n} - \frac{\Delta t}{\Delta x} \left( F_{i+1/2} - F_{i-1/2} \right) \\
                    &= u_{i}^{n} - \frac{\Delta t}{\Delta x} \left[
                    a u_{i}^{n} + \frac{1}{2} a \frac{1-\nu}{\nu}  \left( u_{i}^{n}-u_{i}^{n+1} \right)
                    -\left(
                    a u_{i-1}^{n} + \frac{1}{2} a \frac{1-\nu}{\nu}  \left( u_{i-1}^{n}-u_{i-1}^{n+1} \right)
                \right)
                \right] \\
                    &= u_{i}^{n} - \nu \left( 
                        u_{i}^{n} - u_{i-1}^{n}
                    \right) - \frac{1-\nu}{2} \left( u_{i}^{n}-u_{i}^{n+1} - u_{i-1}^{n} + u_{i-1}^{n+1} \right).
    \end{split}
\end{equation*}

This equation couples the unknowns at time \(t^{n+1}\) into a linear system.
By rearranging the terms to group values at \(t^{n+1}\) on the left and \(t^n\) on the right, we arrive at the remarkably symmetric matrix form of the scheme:
\begin{equation}
    \label{eq:second_order_implicit_matrix}
    \left( 1-\nu \right) u_{i-1}^{n+1} + \left( 1+\nu \right) u_{i}^{n+1} =
    \left( 1+\nu \right) u_{i-1}^{n} + \left( 1-\nu \right) u_{i}^{n}.
\end{equation}
This scheme, sometimes known as the box scheme -- widely used for river modelling, see \cite{morton2005numerical} and references therein -- has several important properties.
First, for a Courant number of \(\nu=1\), the scheme becomes \(2u_i^{n+1} = 2u_{i-1}^n\), or \(u_i^{n+1} = u_{i-1}^n\).
This is the exact solution for advection by one full cell.
This is a significant improvement over the first-order implicit upwind scheme, which becomes less accurate as \(\nu\) increases.

Second, for Courant numbers in the range \(0 \leq \nu \leq 1\), the scheme can be interpreted as a
Reconstruct-Evolve-Average (REA) algorithm.
However, for \(\nu > 1\), the geometric REA interpretation breaks down, even though the scheme remains well-defined and stable.
In this regime, it is best understood purely through its definition as a flux-based method.

The resulting scheme, shown in Equation \eqref{eq:second_order_implicit_matrix}, is a second-order implicit finite volume method.
Interestingly, this formula is algebraically identical to the well-known Box Scheme, which is typically derived from a finite difference perspective.

\subsection{Von Neumann Stability Analysis}
We perform the stability analysis on the scheme for positive velocity \( a > 0 \). The update equation is:
\begin{equation}
    (1-\nu) u_{j-1}^{n+1} + (1+\nu) u_{j}^{n+1} = (1+\nu) u_{j-1}^{n} + (1-\nu) u_{j}^{n}.
\end{equation}
We introduce the Fourier mode \( u_j^n = \hat{u}^n e^{\mathrm{i} j \theta} \), where \( \mathrm{i} = \sqrt{-1} \) and \( \theta = k \Delta x \).
Substituting this into the update equation gives:
\[
    (1-\nu) G e^{\mathrm{i} (j-1) \theta} + (1+\nu) G e^{\mathrm{i} j \theta} = (1+\nu) e^{\mathrm{i} (j-1) \theta} + (1-\nu) e^{\mathrm{i} j \theta}.
\]
Dividing by the common factor \( e^{\mathrm{i} j \theta} \) reduces the terms \( e^{\mathrm{i} (j-1) \theta} \) to \( e^{-\mathrm{i} \theta} \):
\[
    G \left[ (1-\nu) e^{-\mathrm{i} \theta} + (1+\nu) \right] = (1+\nu) e^{-\mathrm{i} \theta} + (1-\nu).
\]
Solving for the amplification factor \( G \):
\[
    G = \frac{(1+\nu) e^{-\mathrm{i} \theta} + (1-\nu)}{(1-\nu) e^{-\mathrm{i} \theta} + (1+\nu)}.
\]
To analyze the magnitude of \( G \), we multiply the numerator and denominator by \( e^{\mathrm{i} \theta/2} \). This symmetrizes the exponents:
\[
    G = \frac{(1+\nu) e^{-\mathrm{i} \theta/2} + (1-\nu) e^{\mathrm{i} \theta/2}}{(1-\nu) e^{-\mathrm{i} \theta/2} + (1+\nu) e^{\mathrm{i} \theta/2}}.
\]
Grouping terms by \(\nu\):
\[
    G = \frac{ (e^{\mathrm{i} \theta/2} + e^{-\mathrm{i} \theta/2}) - \nu(e^{\mathrm{i} \theta/2} - e^{-\mathrm{i} \theta/2}) }{ (e^{\mathrm{i} \theta/2} + e^{-\mathrm{i} \theta/2}) + \nu(e^{\mathrm{i} \theta/2} - e^{-\mathrm{i} \theta/2}) }.
\]
Using Euler's identities \( e^{\mathrm{i} x} + e^{-\mathrm{i} x} = 2\cos x \) and \( e^{\mathrm{i} x} - e^{-\mathrm{i} x} = 2\mathrm{i}\sin x \), this simplifies to:
\begin{equation}
    G = \frac{ 2\cos(\theta/2) - 2\mathrm{i}\nu\sin(\theta/2) }{ 2\cos(\theta/2) + 2\mathrm{i}\nu\sin(\theta/2) } 
      = \frac{ \cos(\theta/2) - \mathrm{i}\nu\sin(\theta/2) }{ \cos(\theta/2) + \mathrm{i}\nu\sin(\theta/2) }.
\end{equation}
The numerator is the complex conjugate of the denominator. 
Therefore, for any real numbers \(\nu\) and \(\theta\), the magnitude is exactly unity:
\begin{equation}
    |G| = \left| \frac{z^*}{z} \right| = 1.
\end{equation}

Consequently, all numerical error for this scheme manifests as dispersion. The numerical phase, \(\phi_{\text{num}} = \arg(G)\), is given by:
\begin{equation}
    \phi_{\text{num}} = -2\arctan\left(\nu\tan\frac{\theta}{2}\right).
    \label{eq:box_scheme_phase}
\end{equation}
An analysis of this phase shows that for Courant numbers \(\nu < 1\), the scheme exhibits a leading phase error (numerical waves travel too fast), while for \(\nu > 1\) it has a lagging phase error (waves travel too slow). For the specific case where \(\nu=1\), the phase is exact, \(\phi_{\text{num}} = -\theta\), confirming that the scheme perfectly advects waves by one cell per time step. This non-dissipative but dispersive character is a key feature of the box scheme, which explains why the solution preserves sharp gradients much better than the first-order upwind method, though it may exhibit trailing oscillations (dispersive ripples) near discontinuities.

\subsection*{General Formulation}
The scheme can be written in a unified form valid for any velocity \( a \) by utilizing the absolute value to automatically select the upwind direction.
The general numerical flux is composed of the standard first-order upwind flux plus an implicit correction term:
\begin{equation}
    \label{eq:second_order_implicit_flux_general}
    \begin{split}
        F_{i+1/2} &= \underbrace{\frac{a+|a|}{2} u_{i}^{n} + \frac{a-|a|}{2} u_{i+1}^{n}}_{\text{Upwind Flux}}  \\
                  &+ 
        \underbrace{\frac{1-|\nu|}{2|\nu|} \left[ \frac{a+|a|}{2} \left( u_{i}^{n} - u_{i}^{n+1} \right) + \frac{a-|a|}{2} \left( u_{i+1}^{n} - u_{i+1}^{n+1} \right) \right]}_{\text{Implicit Correction}}.
    \end{split}
\end{equation}
While algebraically denser, this formulation reveals a key property of the scheme: the correction term vanishes exactly when \( |\nu| = 1 \), reducing the flux to the pure upwind flux of the exact solution. 
Substituting this general flux into the update formula recovers the symmetric matrix coefficients for both positive and negative velocities.

\subsection{The Inflow-Implicit/Outflow-Explicit (IIOE) Scheme}

Similar to the box scheme, we can derive another second-order implicit scheme by defining the reconstruction based on a different set of geometric constraints. For this scheme, we define the reconstruction in cell \(i\) to be the line that passes through two points: the new cell average \(u_i^{n+1}\) at the characteristic origin \(x_i - a\Delta t\), and the old average of the downstream neighbor, \(u_{i+1}^n\), at the location \(x_{i+1}\).

Unlike the box scheme, this reconstruction does not pass through the point \((x_i, u_i^n)\). The slope of this line is given by:
\begin{equation}
    \sigma_i = \frac{u_{i+1}^n - u_i^{n+1}}{\Delta x + a\Delta t}.
    \label{eq:iioe_slope}
\end{equation}
The reconstruction can be written as \(\tilde{u}(x, t^n) = u_i^{n+1} + \sigma_i(x - (x_i - a\Delta t))\). We derive the average flux \(F_{i+1/2}\) by evaluating this reconstruction at the space-time midpoint of the flux-averaging window, \((x_{i+1/2} - a\Delta t/2)\), and multiplying by the advection speed \(a\).
\begin{align*}
    u_{i+1/2, \text{avg}} &= \tilde{u}(x_{i+1/2} - a\Delta t/2) \\
    &= u_i^{n+1} + \sigma_i\left( (x_{i+1/2} - a\Delta t/2) - (x_i - a\Delta t) \right) \\
    &= u_i^{n+1} + \sigma_i\left( (x_{i+1/2} - x_i) + (a\Delta t - a\Delta t/2) \right) \\
    &= u_i^{n+1} + \frac{1}{2}\sigma_i(\Delta x + a\Delta t).
\end{align*}
Substituting the definition of \(\sigma_i\) into this expression yields:
\begin{equation*}
    \begin{split}
        u_{i+1/2, \text{avg}} &= u_i^{n+1} + \frac{1}{2}\left(\frac{u_{i+1}^n - u_i^{n+1}}{\Delta x + a\Delta t}\right)(\Delta x + a\Delta t)\\
                              &= u_i^{n+1} + \frac{1}{2}(u_{i+1}^n - u_i^{n+1}) = \frac{1}{2}(u_{i+1}^n + u_i^{n+1}).
    \end{split}
\end{equation*}
This leads to the remarkably simple and symmetric numerical flux:
\begin{equation}
    F_{i+1/2} = \frac{a}{2}(u_{i+1}^n + u_i^{n+1}).
\end{equation}
This flux can also be written as a correction to the first-order implicit upwind flux:
\begin{equation}
    F_{i+1/2} = \underbrace{a u_i^{n+1}}_{\text{Implicit Upwind}} + \underbrace{\frac{a}{2}(u_{i+1}^n - u_i^{n+1})}_{\text{Correction}}.
\end{equation}
By shifting the index, the inflow flux is \(F_{i-1/2} = \frac{a}{2}(u_i^n + u_{i-1}^{n+1})\). Substituting these into the finite volume update, \(u_i^{n+1} = u_i^n - \frac{\Delta t}{\Delta x}(F_{i+1/2} - F_{i-1/2})\), gives:
\begin{equation}
    u_i^{n+1} = u_i^n - \frac{\nu}{2}\left( (u_{i+1}^n + u_i^{n+1}) - (u_i^n + u_{i-1}^{n+1}) \right).
    \label{eq:iioe_scheme_simple}
\end{equation}
Rearranging this equation to group terms by time level yields the final form of the scheme:
\begin{equation}
    -\frac{\nu}{2}u_{i-1}^{n+1} + \left(1+\frac{\nu}{2}\right)u_i^{n+1} = \left(1+\frac{\nu}{2}\right)u_i^n - \frac{\nu}{2}u_{i+1}^n.
    \label{eq:iioe_scheme}
\end{equation}

\paragraph{Connection to the IIOE Scheme}
This scheme is a second-order accurate implicit method known as the Inflow-Implicit/Outflow-Explicit (IIOE) scheme \cite{2014_Mikula}. It is insightful to note that while our geometric derivation results in both inflow and outflow fluxes being implicit, the scheme's established name originates from an alternative flux-splitting formulation. The algebraic equivalence of these different conceptual paths is a non-trivial result that highlights the rich structure of the scheme.

\paragraph{Properties}
This scheme is fundamentally inconsistent because the underlying reconstruction used to define the fluxes does not satisfy the cell-average constraint, i.e., \(\frac{1}{\Delta x}\int_{x_{i-1/2}}^{x_{i+1/2}} \tilde{u}(x,t^n) dx \neq u_i^n\). This is the same type of inconsistency inherent in the first-order implicit upwind scheme (see Section \ref{sub:implicit_upwind_flux}). A direct consequence is that, unlike the consistent box scheme, the IIOE scheme is not exact when the Courant number is unity (\(\nu=1\)). Despite this inconsistency, the scheme possesses favorable stability properties, which are analyzed in the literature.

\paragraph{Von Neumann Stability Analysis}
To investigate the properties of the scheme given by Equation \eqref{eq:iioe_scheme}, we perform a stability analysis. Substituting the Fourier mode \( u_j^n = \hat{u}^n e^{\mathrm{i} j \theta} \) into the scheme:
\[
    G \left[ \left(1+\frac{\nu}{2}\right) - \frac{\nu}{2}e^{-\mathrm{i}\theta} \right] = \left(1+\frac{\nu}{2}\right) - \frac{\nu}{2}e^{\mathrm{i}\theta}.
\]
Solving for the amplification factor \(G\) gives:
\begin{equation}
    G = \frac{1+\frac{\nu}{2} - \frac{\nu}{2}e^{\mathrm{i}\theta}}{1+\frac{\nu}{2} - \frac{\nu}{2}e^{-\mathrm{i}\theta}}.
    \label{eq:iioe_g}
\end{equation}
An analysis of the magnitude \(|G|\) reveals a fundamental property: the numerator is the complex conjugate of the denominator, i.e., \(N = \bar{D}\).
Therefore, \(|G|^2 = |N|^2 / |D|^2 = |\bar{D}|^2 / |D|^2 = 1\).
This means the IIOE scheme is \textit{unconditionally stable} and \textit{non-dissipative} (\(|G|=1\)), similar to the box scheme.

Consequently, all numerical error for this scheme manifests as dispersion. The numerical phase, \(\phi_{\text{num}} = \arg(G)\), is given by:
\begin{equation}
    \phi_{\text{num}} = -2\arctan\left(\frac{\frac{\nu}{2}\sin\theta}{1+\frac{\nu}{2} - \frac{\nu}{2}\cos\theta}\right) = -2\arctan\left(\frac{\nu\sin\theta}{2+\nu(1-\cos\theta)}\right).
    \label{eq:iioe_phase}
\end{equation}
This phase relationship is nonlinear, indicating the scheme is dispersive. Similar to the box scheme, it can exhibit leading or lagging phase errors depending on the Courant number and wavenumber.

\subsection{A General Blended Implicit Scheme}

Further exploration of this geometric framework reveals that other choices of reconstruction are possible, providing a path to understanding and generalizing families of implicit methods. A notable example is the high-resolution compact scheme developed by Frolkovi\v{c} and \v{Z}erav\'{y} \cite{frolkovic2023high}. While their work utilizes an Inverse Lax-Wendroff procedure, our geometric approach provides an alternative interpretation.

The scheme blends two different second-order accurate fluxes. One component is the IIOE scheme. The other, which can be viewed as an implicit counterpart to the Beam-Warming scheme, is derived from a "mixed-time" slope connecting the current cell center at the old time level to the upwind neighbor at the new time level. The slope is given as:
\[
    \sigma_i = \frac{u_i^n - u_{i-1}^{n+1}}{\Delta x + a\Delta t}.
\]
The corresponding numerical flux, evaluated using the same reconstruction rule as the IIOE scheme (but with this different slope), is:
\[
    F_{i+1/2, 2} = a \left( u_i^{n+1} + \frac{1+\nu}{2}\Delta x \sigma_i \right).
\]
Substituting the slope definition into the flux equation, and using the relation \(\Delta x + a\Delta t = \Delta x(1+\nu)\), reveals a convenient cancellation:
\begin{equation}
    \begin{split}
        F_{i+1/2, 2} &= a \left[ u_i^{n+1} + \frac{1+\nu}{2}\Delta x \left( \frac{u_i^n - u_{i-1}^{n+1}}{\Delta x (1 + \nu)} \right) \right] \\
                  &= a \left[ u_i^{n+1} + \frac{1}{2} \left( u_i^n - u_{i-1}^{n+1} \right) \right].
    \end{split}
\end{equation}

The final blended numerical flux, as presented in \cite{frolkovic2023high}, is then given by:
\begin{equation}
    F_{i+1/2, \omega} = u_{i}^{n+1} - \frac{1}{2}\left( (1-\omega)(u_{i}^{n+1} - u_{i+1}^{n}) + \omega(u_{i-1}^{n+1} - u_{i}^{n})
\right).
\end{equation}
Here, \(\omega \in [0, 1]\) is a blending parameter. When \(\omega=0\), the scheme recovers the IIOE scheme. For a full derivation and detailed analysis of its properties, the reader is referred to the original article.


\section{Comparative Analysis}

A classic von Neumann stability analysis allows us to compare the dissipative (amplitude) and dispersive (phase) errors of the schemes.
The following plots illustrate these properties for a Courant number \(CFL=3.0\).
Figure~\ref{fig:implicit-ampl-analysis} shows the amplification factor magnitude \(|G|\) versus the dimensionless wavenumber \(\theta\), where \(|G|<1\) indicates numerical dissipation.
Figure~\ref{fig:implicit-phase-analysis} shows the relative phase speed, where a value other than 1 indicates a phase error.

\begin{figure}[htbp]
    \centering
    \includestandalone[mode=buildnew]{fig/implicit-ampl-analysis}
    \caption{Comparison of numerical dissipation for the implicit schemes (\(CFL=3.0\)). The second-order Box and IIOE schemes are non-dissipative (\(|G|=1\)), while the first-order implicit upwind scheme is highly dissipative.}
    \label{fig:implicit-ampl-analysis}
\end{figure}

\begin{figure}[htbp]
    \centering
    \includestandalone[mode=buildnew]{fig/implicit-phase-analysis}
    \caption{Comparison of numerical dispersion for the implicit schemes (\(CFL=3.0\)). The black dashed line represents the exact solution.}
    \label{fig:implicit-phase-analysis}
\end{figure}

The analysis reveals significant differences:
\begin{itemize}
    \item The \textit{first-order implicit upwind} scheme is highly dissipative and exhibits substantial phase lag.
    \item The \textit{Box scheme} and \textit{IIOE scheme} are non-dissipative (\(|G|=1\)).
        However, at \(CFL=3.0\), both the Box scheme and the IIOE scheme exhibit lagging phase error.
        These dispersive errors can lead to oscillations in the solution, without damping.
\end{itemize}
For a detailed analysis of the blended scheme of Frolkovi\v{c} and \v{Z}erav\'{y}, the reader is referred to \cite{frolkovic2023high}.

\section{The Implicit Upwind Range Condition}%

In the context of explicit schemes, the Upwind Range Condition (URC) provides a robust geometric constraint for constructing high-resolution methods.
However, in the implicit literature, stability is typically analyzed through algebraic properties of the system matrix, specifically the \textit{M-Matrix property} or \textit{positivity preservation} (see, e.g., Kuzmin \cite{kuzmin2014hierarchical}).

To our knowledge, an extension of the Upwind Range Condition to implicit schemes has not yet been formulated.
In this section, we derive such a condition.
We demonstrate that the algebraic M-Matrix constraints can be reinterpreted geometrically as an \textit{Implicit~Upwind~Range~Condition}.
This unified perspective allows us to construct high-resolution implicit schemes completely analogously to the explicit case.

We begin by analyzing the stability properties of the classical implicit upwind scheme \eqref{eq:implicit_upwind_cons}.

It is well known that this scheme is unconditionally stable for any non-negative Courant number \(\nu \geq 0\).

Observing the coefficients in \eqref{eq:implicit_upwind_solution}, we note that \(\frac{1}{1+\nu} + \frac{\nu}{1+\nu} = 1\).
Furthermore, for any \(\nu \geq 0\), both coefficients are non-negative.
Consequently, \(u_{i}^{n+1}\) is a \textit{convex combination} of the previous local value \(u_{i}^{n}\) and the new upstream value \(u_{i-1}^{n+1}\).

This implies that the solution satisfies a local maximum principle

\[
    \min(u_{i}^{n},u_{i-1}^{n+1}) \leq u_{i}^{n+1} \leq \max(u_{i}^{n},u_{i-1}^{n+1})
\]

By analogy with the explicit case derived in Section \ref{sec:monotonicity_toro}, we can reformulate this boundedness as a constraint:

\begin{equation}
    \label{eq:implicit_upwind_range_condition}
    0 \leq \frac{u_{i}^{n+1}-u_{i}^{n}}{u_{i-1}^{n+1}-u_{i}^{n}} \leq 1.
\end{equation}

Substituting \eqref{eq:implicit_upwind_solution} into \eqref{eq:implicit_upwind_range_condition}, we find that for the first-order implicit upwind scheme, this ratio is exactly:

\[
    \frac{u_{i}^{n+1}-u_{i}^{n}}{u_{i-1}^{n+1}-u_{i}^{n}} = \frac{\nu}{1+\nu}.
\]

Since \(\nu \geq 0\), this ratio is always contained in the interval \([0, 1)\), regardless of the time step size.
This result establishes a direct structural analogy with the Explicit Upwind Range Condition.
In the explicit case, the update is limited by the inflow from the \textit{old} time level (\(u_{i-1}^n - u_i^n\)).
In the implicit case, the update is limited by the inflow from the \textit{new} time level (\(u_{i-1}^{n+1} - u_i^n\)).
We define \eqref{eq:implicit_upwind_range_condition} as the \textit{Implicit Upwind Range Condition} and will utilize it to construct oscillation-free high-resolution implicit schemes.

\paragraph{Connection to Algebraic Constraints}
It is important to note that this geometric condition is the 1D scalar equivalent of the \textit{M-Matrix property} required for monotonicity in general implicit systems, as analyzed by Kuzmin \cite{kuzmin2014hierarchical}.
In the matrix formulation \(A u^{n+1} = u^n\), monotonicity is guaranteed if \(A\) is an M-matrix (i.e., \(A^{-1} \geq 0\)).
While the M-Matrix property provides a rigorous algebraic guarantee of monotonicity, the geometric perspective offered by the Implicit Upwind Range Condition is more intuitive for constructing and understanding the behavior of slope limiters in a finite volume context.
\section{A High-Resolution Box Scheme}
\label{sec:limited_box_scheme}

The analysis has shown that while the box scheme is second-order accurate, it is not monotonic.
To remedy this, we now construct a high-resolution.
We follow an analogous approach to the explicit case.
Since a particular scheme is defined by the slope used for the flux reconstruction, we can limit the slope in such a way to prevent violating the IURC.

\subsection{Slope Limiter}

First note that the box scheme flux \eqref{eq:box-flux-explicit-start} can be expressed also as the implicit upwind flux plus a second-order correction term
\begin{equation}
    \label{eq:box-flux-implicit-start}
    \begin{split}
        F_{i+1/2} &= a u_{i}^{n} + a \frac{1-\nu}{2\nu}  \left( u_{i}^{n}-u_{i}^{n+1} \right)\\
                  &= a u_{i}^{n+1} + a\frac{1+\nu}{2\nu}\left( u_{i}^{n}-u_{i}^{n+1} \right).
    \end{split}
\end{equation}

As before, we introduce \(\Psi_i\) to limit the slope
\[
    F_{i+1/2}^{\text{lim}} = a u_{i}^{n+1} + a\frac{1+\nu}{2\nu}\left( u_{i}^{n}-u_{i}^{n+1} \right)\Psi_i.
\]
Notice that we have the unconditionally stable implicit upwind flux as a safety net with \(\Psi_i = 0\).
The choice \(\Psi_i = 1\) reproduces the box scheme.
In the next section we derive conditions for the slope-limiter in order to satisfy the IURC.

\subsection{Derivation of the Limiter Constraint}
\label{sec:implicit_limiter_derivation}

Substituting the limited fluxes to the update formula we get
\[
u_i^{n+1} = u_{i}^{n} -\nu(u_i^{n+1} - u_{i-1}^{n+1}) - \frac{1+\nu}{2} \left[ \Psi_i(u_i^n - u_i^{n+1}) - \Psi_{i-1}(u_{i-1}^n - u_{i-1}^{n+1}) \right]
\]
To derive conditions for the limiters, we want to enforce the IURC \eqref{eq:implicit_upwind_range_condition} to the limited box scheme.
After some algebraic manipulation and using
\[
    u_{i}^{n+1} - u_{i-1}^{n+1} = u_{i}^{n+1} - u_{i}^{n} - \left( u_{i-1}^{n+1} - u_{i}^{n} \right)
\]
we arrive at the equation
\begin{multline*}
    u_{i}^{n+1}-u_{i}^{n} =
    -\nu\left( u_{i}^{n+1}-u_{i}^{n} \right) + \nu \left( u_{i-1}^{n+1}-u_{i}^{n} \right)\\
    -\frac{1+\nu}{2}
    \left[ \Psi_i \left( u_{i}^{n}-u_{i}^{n+1} \right) - \Psi_{i-1} \left( u_{i-1}^{n}-u_{i-1}^{n+1} \right) \right].
\end{multline*}
Assuming \(u_{i}^{n+1} - u_{i}^{n} \neq 0\) and defining
\[
    r_i = \frac{u_{i-1}^{n}-u_{i-1}^{n+1}}{u_{i}^{n}-u_{i}^{n+1}}
\]
we arrive at (notice the change of the sign before the limiter terms)
\[
    1 = -\nu + \nu\frac{u_{i-1}^{n+1}-u_{i}^{n}}{u_{i}^{n+1}-u_{i}^{n}}
    +\frac{1+\nu}{2} \left( \Psi_i - \Psi_{i-1}r_i \right).
\]
Note the definition of \(r_i\).
Unlike in the explicit case where it is a ratio of gradients computed from old values, here it is a ratio of the implicit gradients defined earlier.
This reformulation is key to extending the limiter framework to the implicit context.
We can recognize the reciprocal of the ratio in IURC in the second term on the RHS.
To satisfy the IURC we need
\[
    0 \leq \frac{u_{i}^{n+1}-u_{i}^{n}}{u_{i-1}^{n+1}-u_{i}^{n}} \leq 1
    \implies \frac{u_{i-1}^{n+1}-u_{i}^{n}}{u_{i}^{n+1}-u_{i}^{n}} \geq 1.
\]
Thus, we deduce the constraint on the limiter
\begin{equation}
    \begin{split}
        1 - \frac{1+\nu}{2} \left( \Psi_i - \Psi_{i-1}r_i \right) &\geq 0\\
        \Psi_i - \Psi_{i-1}r_i &\leq \frac{2}{1+\nu}.
    \end{split}
\end{equation}
This inequality generalizes the classical Sweby TVD constraints \cite{sweby1984high} to the implicit Box scheme framework, where the bound is now scaled by the Courant number factor \(\frac{1}{1+\nu}\).
This condition defines the maximum allowable value for \(\Psi_i\) to maintain monotonicity, demonstrating how the IURC directly constrains the limiter depending on the local ratio of implicit gradients changes \(r_i\), and the Courant number \(\nu\).

\subsection{A sufficient condition}%
\label{sub:implicit_limiter_suff}

Defining lower and upper bounds for the limiter, we can also derive a sufficient condition, just as before.
Let
\[
    u_{i}^{n+1} \leq u_{i}^{n+1} + \frac{1+\nu}{2\nu} \left( u_{i}^{n}-u_{i}^{n+1} \right) \Psi_i \leq u_{i+1}^{n}.
\]
We restrict the flux from below by the implicit upwind and above by unlimited box scheme
\[
    0 \leq \Psi_i \leq 1.
\]
Assuming \(r_i>0\)(monotone section), a sufficient condition for the limiter is 
\begin{equation*}
    \begin{split}
        \frac{\Psi_i}{r_i} &\leq \frac{2}{(1+\nu)r_i} - \Psi_{i-1}\\
        \frac{\Psi_i}{r_i} &\leq \frac{2}{(1+\nu)r_i} - 1\\
        \frac{\Psi_i}{r_i} &\leq \frac{2 - (1+\nu)r_i}{(1+\nu)r_i}
    \end{split}
\end{equation*}

While the sufficient condition guarantees monotonicity, it is often overly restrictive, leading to excessive numerical diffusion.
As in \cite{frolkovic2023high}, we will solve the resulting system of equations by sweeping.
This way we can utilize the necessary condition, in which the limiter from the neighbor cell \(\Psi_{i-1}\) relaxes the condition on the limiter so that we can have a sharper solution.
