%! TeX root: ../main.tex
\chapter{The Implicit Geometric Framework}%
\label{sec:The Implicit Geometric Framework}

The explicit finite volume schemes discussed in the previous chapter are limited by the Courant-Friedrichs-Lewy (CFL) condition, requiring \(0 \leq \nu \leq 1\).
For many practical applications~--~such as flows with varying velocities or grids with variable sizes~--~this restriction forces the use of impractically small time steps.
To overcome this barrier, we turn to \textit{semi-implicit} methods.

In this chapter, we extend the geometric arguments established for explicit methods to construct high-order semi-implicit schemes.
We demonstrate that the monotonicity constraints derived in the explicit framework have direct analogues in the implicit setting.
We begin by analyzing the classical first-order implicit upwind scheme to establish the baseline for stability and accuracy, before deriving our high-resolution semi-implicit extensions.

\section{The First-Order Implicit Upwind Scheme}
\label{sec:implicit_upwind}
Let us consider the scheme
\begin{equation}
    \label{eq:implicit-upwind-cons}
    u_{i}^{n+1} = u_{i}^{n} - \nu\left(u_{i}^{n+1} - u_{i-1}^{n+1}\right).
\end{equation}
This is often referred to as the \textit{implicit upwind} or \textit{backward-time backward-space (BTBS)} scheme.
Unlike explicit methods, the values from the new time step \(t^{n+1}\) appear on both sides of the equation, coupling the grid points into a linear system.
It has the remarkable property, that it is stable for \(\nu \geq 0\).

The cost of the stability is considerable numerical diffusion. Compared to the explicit upwind scheme, the implicit formulation exhibits significantly higher damping.
Another notable difference is the relationship between accuracy and the time step: while the explicit upwind scheme becomes exact as the Courant number approaches 1, the implicit scheme becomes less accurate as the Courant number increases.
We analyze this behavior using geometric arguments.

By rearranging \eqref{eq:implicit-upwind-cons} we solve for the new average \(u_{i}^{n+1}\):
\begin{equation}
    \label{eq:implicit_upwind_solution}
    u_{i}^{n+1} = \frac{1}{1+\nu}u_{i}^{n} + \frac{\nu}{1+\nu}u_{i-1}^{n+1}.
\end{equation}
If we supply inflow boundary conditions and sweep from the upwind direction, \(u_{i-1}^{n+1}\) is already known when computing \(u_{i}^{n+1}\).
Thus, the only unknown in equation \eqref{eq:implicit_upwind_solution} is \(u_{i}^{n+1}\).

We observe that the solution \(u_{i}^{n+1}\) is a linear interpolation between \(u_{i}^{n}\) and \(u_{i-1}^{n+1}\).
In contrast, the explicit scheme is a linear interpolation between the old values \(u_{i}^{n}\) and \(u_{i-1}^{n}\) (see Equation \eqref{eq:godunov_integral}).

To gain more insight, we visualize the old values along with the new values at the old time level \(t^n\).
Since the solution travels along characteristics, we can trace back from the new time level to the old one.
The boundaries from which the new average \(u_{i}^{n+1}\) is computed correspond to the boundaries of the grid cell \(i\) shifted backwards by \(a\Delta t\).

\begin{figure}[h]
    \centering

    \begin{subfigure}[b]{\textwidth}
        \centering
        \includestandalone[mode=buildnew]{fig/backtrace}
        %\caption{}
    \end{subfigure}

    \vspace{1cm}
    
    \begin{subfigure}[b]{\textwidth}
        \centering
        \includestandalone[mode=buildnew]{fig/backtrace-front}
        %\caption{}
    \end{subfigure}

    \caption{Visualization of the backward characteristic tracing.
        (Top) Characteristics are traced backward from \(t^{n+1}\) to \(t^n\).
        (Bottom) The implicit reconstruction using values from the new time level.
        Note that the value \(u_i^{n+1}\) originates from the upstream location shifted by \(a\Delta t\).}
    \label{fig:backtrace}
\end{figure}

As illustrated in Figure \ref{fig:backtrace}, this geometric view explains the opposing accuracy behavior.
In the implicit case, as the time step (and thus \(\nu\)) increases, the value \(u_{i-1}^{n+1}\) gets further away from \(u_{i-1}^{n}\).
In contrast, for the explicit upwind scheme, as \(\nu \to 1\), the computed new average is better approximated by the upwind neighbor \(u_{i-1}^{n}\).

\subsection{Geometric Interpretation: Flux Formulation}%
\label{sub:implicit_upwind_geometric_flux}

Unlike the explicit Godunov scheme, the implicit upwind scheme cannot be strictly interpreted as evolving a reconstruction of the known data \(u^n\).
Instead, it is best understood through the geometric definition of the \textit{numerical flux}.

Recall that the finite volume update requires the time-averaged flux across the interface:
\[
    F_{i-1/2} \approx \frac{1}{\Delta t} \int_{t_n}^{t_{n+1}} f(u(x_{i-1/2}, t)) \, \mathrm{d}t.
\]
The implicit case differs in the flux approximation.
Specifically, we assume a piecewise constant profile determined by the new averages:
\[
    \tilde{u}(x,t^n) = u_{i}^{n+1}.
\]
This reveals a critical property of the implicit formulation: it employs a reconstruction that is inconsistent with the data at time \(t^n\).
Specifically, it does not satisfy the integral average constraint:
\[
    \frac{1}{\Delta x}\int\limits_{{x_{i-1/2}}}^{{x_{i+1/2}}} {\tilde{u}(x,t^n)} \: \mathrm{d}{x} = u_{i}^{n+1} \neq u_{i}^{n}.
\]

The numerical flux \(F_{i-1/2}\) is defined as the geometric area of this \textit{future} profile swept across the interface during the time step \(\Delta t\).
For the upwind case (\(a>0\)), the flow comes from the left, so the flux is determined by the reconstruction in cell \(i-1\):
\begin{equation}
    \label{eq:implicit_upwind_flux}
    F_{i-1/2} = a \cdot \tilde{u}(x_{i-1/2}^-) = a u_{i-1}^{n+1}.
\end{equation}

\begin{figure}[ht]
    \centering
    \includestandalone{fig/implicit-upwind-reconstruct}
    \caption{Reconstruction step of the implicit upwind scheme \eqref{eq:implicit-upwind-cons}.}
    \label{fig:implicit_upwind_reconstruct}
\end{figure}

This highlights that in case of the implicit upwind scheme, the reconstruction serves solely to define the \textit{fluxes}, not to represent the evolution of the specific fluid elements currently in the cell.
\section{The Implicit Upwind Range Condition}%
\label{sec:implicit-upwind-range}

In the context of explicit schemes, the Upwind Range Condition (URC) provides a robust geometric constraint for constructing high-resolution methods.
However, in the implicit literature, stability is typically analyzed through algebraic properties of the system matrix, specifically the \textit{M-Matrix property} or \textit{positivity preservation} (see, e.g., Kuzmin \cite{kuzmin2014hierarchical}).

To our knowledge, an extension of the Upwind Range Condition to implicit schemes has not yet been formulated.
In this section, we derive such a condition.
We demonstrate that the algebraic M-Matrix constraints can be reinterpreted geometrically as an \textit{Implicit Upwind Range Condition}.
This unified perspective allows us to construct high-resolution semi-implicit schemes completely analogously to the explicit case.

We begin by analyzing the stability properties of the classical implicit upwind scheme:

\begin{equation}
    \label{eq:implicit-upwind}
    u_{i}^{n+1} = u_{i}^{n} - \nu\left(u_{i}^{n+1} - u_{i-1}^{n+1}\right),
\end{equation}

It is well known that this scheme is unconditionally stable for any non-negative Courant number \(\nu \geq 0\).
This stability can be understood by rearranging \eqref{eq:implicit-upwind} to solve for the new average \(u_{i}^{n+1}\):

\begin{equation}
    \label{eq:implicit-upwind-solution}
    u_{i}^{n+1} = \frac{1}{1+\nu}u_{i}^{n} + \frac{\nu}{1+\nu}u_{i-1}^{n+1}.
\end{equation}

Observing the coefficients, we note that \(\frac{1}{1+\nu} + \frac{\nu}{1+\nu} = 1\).
Furthermore, for any \(\nu \geq 0\), both coefficients are non-negative.
Consequently, \(u_{i}^{n+1}\) is a \textit{convex combination} of the previous local value \(u_{i}^{n}\) and the new upstream value \(u_{i-1}^{n+1}\).

This implies that the solution satisfies a local maximum principle

\[
    \min(u_{i}^{n},u_{i-1}^{n+1}) \leq u_{i}^{n+1} \leq \max(u_{i}^{n},u_{i-1}^{n+1})
\]

By analogy with the explicit case derived in Section \ref{sec:monotonicity_toro}, we can reformulate this boundedness as a constraint:

\begin{equation}
    \label{eq:implicit-upwind-range-condition}
    0 \leq \frac{u_{i}^{n+1}-u_{i}^{n}}{u_{i-1}^{n+1}-u_{i}^{n}} \leq 1.
\end{equation}

Substituting \eqref{eq:implicit-upwind-solution} into \eqref{eq:implicit-upwind-range-condition}, we find that for the first-order implicit upwind scheme, this ratio is exactly:

\[
    \frac{u_{i}^{n+1}-u_{i}^{n}}{u_{i-1}^{n+1}-u_{i}^{n}} = \frac{\nu}{1+\nu}.
\]

Since \(\nu \geq 0\), this ratio is always contained in the interval \([0, 1)\), regardless of the time step size.
This result establishes a direct structural analogy with the Explicit Upwind Range Condition.
In the explicit case, the update is limited by the inflow from the \textit{old} time level (\(u_{i-1}^n - u_i^n\)).
In the implicit case, the update is limited by the inflow from the \textit{new} time level (\(u_{i-1}^{n+1} - u_i^n\)).
We define \eqref{eq:implicit-upwind-range-condition} as the \textit{Implicit Upwind Range Condition} and will utilize it to construct oscillation-free high-resolution semi-implicit schemes.

\paragraph{Connection to Algebraic Constraints}
It is important to note that this geometric condition is the 1D scalar equivalent of the \textit{M-Matrix property} required for monotonicity in general implicit systems, as analyzed by Kuzmin \cite{kuzmin2014hierarchical}.
In the matrix formulation \(A u^{n+1} = u^n\), monotonicity is guaranteed if \(A\) is an M-matrix (i.e., \(A^{-1} \geq 0\)).
