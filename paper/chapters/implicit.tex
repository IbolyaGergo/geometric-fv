%near start,! TeX root: ../main.tex
\chapter{The Implicit Geometric Framework}%
\label{sec:The Implicit Geometric Framework}

The explicit finite volume schemes discussed in the previous chapter are limited by the Courant-Friedrichs-Lewy (CFL) condition, requiring \(0 \leq \nu \leq 1\).
For many practical applications~--~such as flows with varying velocities or grids with variable sizes~--~this restriction forces the use of impractically small time steps.
To overcome this barrier, we turn to \textit{implicit} methods.

In this chapter, we extend the geometric arguments established for explicit methods to construct high-order implicit schemes.
We demonstrate that the monotonicity constraints derived in the explicit framework have direct analogues in the implicit setting.
We begin by analyzing the classical first-order implicit upwind scheme to establish the baseline for stability and accuracy, before deriving our high-resolution implicit extensions.

\section{The First-Order Implicit Upwind Scheme}
\label{sec:implicit_upwind}
Let us consider the scheme
\begin{equation}
    \label{eq:implicit_upwind_cons}
    u_{i}^{n+1} = u_{i}^{n} - \nu\left(u_{i}^{n+1} - u_{i-1}^{n+1}\right).
\end{equation}
This is often referred to as the \textit{implicit upwind} or \textit{backward-time backward-space (BTBS)} scheme.
Unlike explicit methods, the values from the new time step \(t^{n+1}\) appear on both sides of the equation, coupling the grid points into a linear system.
It has the remarkable property, that it is stable for \(\nu \geq 0\).

% TODO: Plot explicit upwind vs implicit upwind to show the difference

By rearranging \eqref{eq:implicit_upwind_cons} we solve for the new average \(u_{i}^{n+1}\):
\begin{equation}
    \label{eq:implicit_upwind_solution}
    u_{i}^{n+1} = \frac{1}{1+\nu}u_{i}^{n} + \frac{\nu}{1+\nu}u_{i-1}^{n+1}.
\end{equation}
If we supply inflow boundary conditions and sweep from the upwind direction, \(u_{i-1}^{n+1}\) is already known when computing \(u_{i}^{n+1}\).
Thus, the only unknown in equation \eqref{eq:implicit_upwind_solution} is \(u_{i}^{n+1}\).

We observe that the solution \(u_{i}^{n+1}\) is a linear interpolation between \(u_{i}^{n}\) and \(u_{i-1}^{n+1}\).
In contrast, the explicit scheme is a linear interpolation between the old values \(u_{i}^{n}\) and \(u_{i-1}^{n}\) (see Equation \eqref{eq:godunov_integral}).

\subsection{Von Neumann Stability Analysis}
To formally verify the stability of the scheme, we perform a von Neumann stability analysis. 
We consider the evolution of a single Fourier mode. 
To avoid confusion with the imaginary unit \( \mathrm{i} = \sqrt{-1} \), we use the index \( j \) for spatial position in this subsection:
\[
    u_j^n = \hat{u}^n e^{\mathrm{i} k j \Delta x},
\]
where \(k\) is the wave number.
The solution at the next time step is related by an amplification factor \(G\), such that \(u_j^{n+1} = G \hat{u}^n e^{\mathrm{i} k j \Delta x}\). 
Substituting this into the scheme \eqref{eq:implicit_upwind_cons} yields:
\begin{align*}
    G \hat{u}^n e^{\mathrm{i} k j \Delta x} &= \hat{u}^n e^{\mathrm{i} k j \Delta x} - \nu \left( G \hat{u}^n e^{\mathrm{i} k j \Delta x} - G \hat{u}^n e^{\mathrm{i} k (j-1) \Delta x} \right).
\end{align*}
Dividing by \(\hat{u}^n e^{\mathrm{i} k j \Delta x}\) and rearranging gives:
\begin{align*}
    G &= 1 - \nu G (1 - e^{-\mathrm{i} k \Delta x}) \\
    G (1 + \nu(1 - e^{-\mathrm{i} k \Delta x})) &= 1.
\end{align*}
Solving for the amplification factor \(G\) results in:
\begin{equation}
    \label{eq:implicit_upwind_g}
    G = \frac{1}{1 + \nu(1 - e^{-\mathrm{i}\theta})},
\end{equation}
where \(\theta = k \Delta x\) is the dimensionless wave number. 
To analyze stability, we examine the magnitude of \(G\). The squared magnitude of the denominator is:
\begin{align*}
    |1 + \nu(1 - \cos\theta + \mathrm{i}\sin\theta)|^2 &= |(1 + \nu(1-\cos\theta)) + \mathrm{i}\nu\sin\theta|^2 \\
    &= (1 + \nu(1-\cos\theta))^2 + (\nu\sin\theta)^2 \\
    &= 1 + 2\nu(1-\cos\theta) + \nu^2(1-\cos\theta)^2 + \nu^2\sin^2\theta \\
    &= 1 + 2\nu(1-\cos\theta) + \nu^2(1 - 2\cos\theta + \cos^2\theta + \sin^2\theta) \\
    &= 1 + 2\nu(1-\cos\theta) + 2\nu^2(1 - \cos\theta) \\
    &= 1 + 2(\nu + \nu^2)(1-\cos\theta).
\end{align*}
For any Courant number \(\nu \geq 0\), the term \(2(\nu + \nu^2)\) is non-negative. 
The term \(1-\cos\theta\) is also always non-negative. 
Therefore, the denominator's squared magnitude is always greater than or equal to 1. 
Since \(|G|^2 = 1 / |\text{denominator}|^2\), we have:
\[
    |G|^2 \leq 1 \quad \text{for all } \nu \geq 0 \text{ and for all } \theta.
\]
This proves that the first-order implicit upwind scheme is \textit{unconditionally stable}.

The magnitude \(|G|\) also reveals the scheme's dissipative nature. 
The scheme is most dissipative for the highest-frequency waves representable on the grid, where \(\theta = \pi\). 
At this frequency, \( \cos \theta = -1 \), and the denominator squared becomes \( 1 + 4\nu + 4\nu^2 = (1+2\nu)^2 \). 
Consequently, \(|G| = 1/(1+2\nu)\). 
As \(\nu\) increases, this value decreases, meaning the scheme strongly damps short-wavelength oscillations. 
While this ensures robustness, it is the cause of the significant numerical diffusion observed in solutions.

\subsection{Geometric and Physical Interpretation}%
\label{sub:implicit_upwind_geom_interp}

To gain more insight, we visualize the old values along with the new values at the old time level \(t^n\).
Since the solution travels along characteristics, we can trace back from the new time level to the old one.
The boundaries from which the new average \(u_{i}^{n+1}\) is computed correspond to the boundaries of the grid cell \(i\) shifted backwards by \(a\Delta t\).

\begin{figure}[htbp]
    \centering

    \begin{subfigure}[b]{\textwidth}
        \centering
        \includestandalone[mode=buildnew]{fig/backtrace}
        %\caption{}
    \end{subfigure}

    \vspace{1cm}
    
    \begin{subfigure}[b]{\textwidth}
        \centering
        \includestandalone[mode=buildnew]{fig/backtrace-front}
        %\caption{}
    \end{subfigure}

    \caption{Visualization of the backward characteristic tracing.
        (Top) Characteristics are traced backward from \(t^{n+1}\) to \(t^n\).
        (Bottom) The implicit reconstruction using values from the new time level.
        Note that the value \(u_i^{n+1}\) originates from the upstream location shifted by \(a\Delta t\).}
    \label{fig:backtrace}
\end{figure}

As illustrated in Figure \ref{fig:backtrace}, this geometric view explains the opposing accuracy behavior.
In the implicit case, as the time step (and thus \(\nu\)) increases, the value \(u_{i-1}^{n+1}\) becomes a less accurate approximation of the inflow than \(u_{i-1}^{n}\).
In contrast, for the explicit upwind scheme, as \(\nu \to 1\), the inflow is better approximated by the upwind neighbor \(u_{i-1}^{n}\).

\subsection{Numerical Flux and Reconstruction Inconsistency}%
\label{sub:implicit_upwind_flux}

Unlike the explicit scheme, the implicit upwind scheme cannot be strictly interpreted as evolving a reconstruction of the data at \(t^n\).
Instead, it is best understood through the geometric definition of the \textit{numerical flux}.

Recall that the finite volume update requires the time-averaged flux across the interface:
\[
    F_{i+1/2} \approx \frac{1}{\Delta t} \int_{t_n}^{t_{n+1}} f(u(x_{i+1/2}, t)) \, \diff t.
\]
The implicit case differs in the approximation of this integral.
Specifically, to estimate the fluxes, we assume a piecewise constant profile determined by the \textit{new} averages:
\[
    \tilde{u}(x,t^n) = u_{i}^{n+1}.
\]
For positive velocity \(a > 0\), the flow across the interface \(x_{i+1/2}\) is determined by the upwind cell \(i\). Hence, the numerical flux is:
\begin{equation}
    \label{eq:implicit-upwind-flux}
    F_{i+1/2} = a u_{i}^{n+1}.
\end{equation}

\begin{figure}[ht]
    \centering

    \begin{subfigure}[b]{\textwidth}
        \centering
        \includestandalone[mode=buildnew]{fig/implicit-upwind-reconstruct}
    \end{subfigure}

    \vspace{1cm}
    
    \begin{subfigure}[b]{\textwidth}
        \centering
        \includestandalone[mode=buildnew]{fig/implicit-upwind-flux}
    \end{subfigure}

    \caption{(Top) The piecewise-constant data profile \(\tilde{u}(x,t^n)\) used to define the numerical flux in the implicit upwind scheme \eqref{eq:implicit_upwind_cons}.
    (Bottom) The numerical flux is the area swept (red) by this reconstruction across the interfaces during the evolution.}
    \label{fig:implicit_upwind_flux}
\end{figure}

This reveals a critical property of the implicit upwind scheme: it employs a reconstruction that is inconsistent with the data at time \(t^n\).
Specifically, the cell average of the reconstruction is not the cell average of the data at time \(t^n\):
\[
    \frac{1}{\Delta x}\int\limits_{{x_{i-1/2}}}^{{x_{i+1/2}}} {\tilde{u}(x,t^n)} \,\diff x = u_{i}^{n+1} \neq u_{i}^{n}.
\]

This inconsistency is a primary source of the scheme's low accuracy and high numerical diffusion.
In case of the implicit upwind scheme, the reconstruction serves solely to define the \textit{fluxes}, not to represent the evolution of the specific fluid elements currently in the cell.

To overcome these limitations, we must develop higher-order schemes based on more sophisticated, consistent reconstructions, which we explore in the next section.

\section{Second-Order implicit schemes}%
\label{sec:second_order_implicit}

% TODO: Plot result second order vs implicit upwind.
%"Caption: "Comparison of numerical solutions for the advection of a Gaussian pulse.
% The first-order scheme exhibits significant numerical diffusion, while the second-order scheme remains much closer to the exact solution.
To construct a second-order scheme, we replace the piecewise-constant profile with a piecewise-linear one.
A key challenge is defining an appropriate slope, \(\sigma_i\), for each cell.
To construct an implicit scheme, we can use the yet-unknown value \(u_i^{n+1}\) to define the slope.
We define the slope such that the linear profile at time \(t^n\), when evaluated at \(x_i\), is equal to the old value \(u_{i}^{n}\), and when it is evaluated at the back-traced position \(x_i - a\Delta t\), is equal to the new cell average \(u_i^{n+1}\).
This yields a reconstruction:
\[
    \tilde{u}(x, t^n) = u_i^n + \sigma_i (x - x_i).
\]
For this to hold, we require \(\tilde{u}(x_i - a\Delta t, t^n) = u_i^{n+1}\), which gives:
\[
    u_i^n + \sigma_i (x_i - a\Delta t - x_i) = u_i^{n+1} \implies \sigma_i = \frac{ u_{i}^{n}-u_{i}^{n+1} }{a\Delta t}.
\]
This choice of slope provides a direct physical link between the spatial variation within the cell and the temporal change of its average value.
Crucially, since the reconstruction has a value of \(u_i^n\) at the cell's midpoint, it satisfies the integral average constraint, resolving the inconsistency of the first-order scheme.
This reconstruction is illustrated in Figure~\ref{fig:box_reconstruct}. 
\begin{figure}[htbp]
    \centering

    \begin{subfigure}[b]{\textwidth}
        \centering
        \includestandalone[mode=buildnew]{fig/box-reconstruct}
    \end{subfigure}

    \vspace{1cm}
    
    \begin{subfigure}[b]{\textwidth}
        \centering
        \includestandalone[mode=buildnew]{fig/box-flux}
    \end{subfigure}

    \caption{(Top) Second-order implicit, piecewise-linear reconstruction at time \(t^n\).
        The slope \(\sigma_i\) is chosen such that the line passes through the old average \(u_{i}^{n}\) (black dot) at the cell center and the new average \(u_i^{n+1}\) (green dot) at the back-traced location \(x_i - a\Delta t\).
(Bottom) Numerical flux for the second-order implicit scheme.
The red trapezoids represent the area swept across the cell interfaces by the linear reconstruction over the time step.}
    \label{fig:box_reconstruct}
\end{figure}

The linear reconstruction (see Section \ref{sub:second_order}) yields a flux
\begin{equation}
    \begin{split}
        F_{i+1/2} &= a u_i^n + \frac{1}{2} a (1 - \nu) \Delta x \sigma_i \\
                  &= a u_{i}^{n} + \frac{1}{2} a (1-\nu) \Delta x \frac{ u_{i}^{n}-u_{i}^{n+1} }{a\Delta t} \\
                  &= a u_{i}^{n} + \frac{1}{2} a \frac{1-\nu}{\nu}  \left( u_{i}^{n}-u_{i}^{n+1} \right)
    \end{split}
\end{equation}
A corresponding expression can be written for the flux \(F_{i-1/2}\) at the left-hand interface by shifting the index from \(i\) to \(i-1\).


Substituting these fluxes into the general finite volume update formula, Equation \eqref{eq:fvm_general}, gives the update formula for the new scheme:
\begin{equation*}
    \begin{split}
        u_{i}^{n+1} &= u_{i}^{n} - \frac{\Delta t}{\Delta x} \left( F_{i+1/2} - F_{i-1/2} \right) \\
                    &= u_{i}^{n} - \frac{\Delta t}{\Delta x} \left[
                    a u_{i}^{n} + \frac{1}{2} a \frac{1-\nu}{\nu}  \left( u_{i}^{n}-u_{i}^{n+1} \right)
                    -\left(
                    a u_{i-1}^{n} + \frac{1}{2} a \frac{1-\nu}{\nu}  \left( u_{i-1}^{n}-u_{i-1}^{n+1} \right)
                \right)
                \right] \\
                    &= u_{i}^{n} - \nu \left( 
                        u_{i}^{n} - u_{i-1}^{n}
                    \right) - \frac{1-\nu}{2} \left( u_{i}^{n}-u_{i}^{n+1} - u_{i-1}^{n} + u_{i-1}^{n+1} \right).
    \end{split}
\end{equation*}

This equation couples the unknowns at time \(t^{n+1}\) into a linear system.
By rearranging the terms to group values at \(t^{n+1}\) on the left and \(t^n\) on the right, we arrive at the remarkably symmetric matrix form of the scheme:
\begin{equation}
    \label{eq:second_order_implicit_matrix}
    \left( 1-\nu \right) u_{i-1}^{n+1} + \left( 1+\nu \right) u_{i}^{n+1} =
    \left( 1+\nu \right) u_{i-1}^{n} + \left( 1-\nu \right) u_{i}^{n}.
\end{equation}
This scheme, sometimes known as the box scheme -- widely used for river modelling, see \cite{morton2005numerical} and references therein -- has several important properties.
First, for a Courant number of \(\nu=1\), the scheme becomes \(2u_i^{n+1} = 2u_{i-1}^n\), or \(u_i^{n+1} = u_{i-1}^n\).
This is the exact solution for advection by one full cell.
This is a significant improvement over the first-order implicit upwind scheme, which becomes less accurate as \(\nu\) increases.

Second, for Courant numbers in the range \(0 \leq \nu \leq 1\), the scheme can be interpreted as a
Reconstruct-Evolve-Average (REA) algorithm.
However, for \(\nu > 1\), the geometric REA interpretation breaks down, even though the scheme remains well-defined and stable.
In this regime, it is best understood purely through its definition as a flux-based method.

The resulting scheme, shown in Equation \eqref{eq:second_order_implicit_matrix}, is a second-order implicit finite volume method.
Interestingly, this formula is algebraically identical to the well-known Box Scheme, which is typically derived from a finite difference perspective.

\subsection*{General Formulation}
The scheme can be written in a unified form valid for any velocity \( a \) by utilizing the absolute value to automatically select the upwind direction.
The general numerical flux is composed of the standard first-order upwind flux plus an implicit correction term:
\begin{equation}
    \label{eq:second_order_implicit_flux_general}
    \begin{split}
        F_{i+1/2} &= \underbrace{\frac{a+|a|}{2} u_{i}^{n} + \frac{a-|a|}{2} u_{i+1}^{n}}_{\text{Upwind Flux}}  \\
                  &+ 
        \underbrace{\frac{1}{2}\frac{1-|\nu|}{|\nu|} \left[ \frac{a+|a|}{2} \left( u_{i}^{n} - u_{i}^{n+1} \right) + \frac{a-|a|}{2} \left( u_{i+1}^{n} - u_{i+1}^{n+1} \right) \right]}_{\text{Implicit Correction}}.
    \end{split}
\end{equation}
While algebraically denser, this formulation reveals a key property of the scheme: the correction term vanishes exactly when \( |\nu| = 1 \), reducing the flux to the pure upwind flux of the exact solution. 
Substituting this general flux into the update formula recovers the symmetric matrix coefficients for both positive and negative velocities.

\section{The Implicit Upwind Range Condition}%
\label{sec:implicit-upwind-range}

In the context of explicit schemes, the Upwind Range Condition (URC) provides a robust geometric constraint for constructing high-resolution methods.
However, in the implicit literature, stability is typically analyzed through algebraic properties of the system matrix, specifically the \textit{M-Matrix property} or \textit{positivity preservation} (see, e.g., Kuzmin \cite{kuzmin2014hierarchical}).

To our knowledge, an extension of the Upwind Range Condition to implicit schemes has not yet been formulated.
In this section, we derive such a condition.
We demonstrate that the algebraic M-Matrix constraints can be reinterpreted geometrically as an \textit{Implicit Upwind Range Condition}.
This unified perspective allows us to construct high-resolution implicit schemes completely analogously to the explicit case.

We begin by analyzing the stability properties of the classical implicit upwind scheme \eqref{eq:implicit_upwind_cons}.

It is well known that this scheme is unconditionally stable for any non-negative Courant number \(\nu \geq 0\).

Observing the coefficients in \eqref{eq:implicit_upwind_solution}, we note that \(\frac{1}{1+\nu} + \frac{\nu}{1+\nu} = 1\).
Furthermore, for any \(\nu \geq 0\), both coefficients are non-negative.
Consequently, \(u_{i}^{n+1}\) is a \textit{convex combination} of the previous local value \(u_{i}^{n}\) and the new upstream value \(u_{i-1}^{n+1}\).

This implies that the solution satisfies a local maximum principle

\[
    \min(u_{i}^{n},u_{i-1}^{n+1}) \leq u_{i}^{n+1} \leq \max(u_{i}^{n},u_{i-1}^{n+1})
\]

By analogy with the explicit case derived in Section \ref{sec:monotonicity_toro}, we can reformulate this boundedness as a constraint:

\begin{equation}
    \label{eq:implicit_upwind_range_condition}
    0 \leq \frac{u_{i}^{n+1}-u_{i}^{n}}{u_{i-1}^{n+1}-u_{i}^{n}} \leq 1.
\end{equation}

Substituting \eqref{eq:implicit_upwind_solution} into \eqref{eq:implicit_upwind_range_condition}, we find that for the first-order implicit upwind scheme, this ratio is exactly:

\[
    \frac{u_{i}^{n+1}-u_{i}^{n}}{u_{i-1}^{n+1}-u_{i}^{n}} = \frac{\nu}{1+\nu}.
\]

Since \(\nu \geq 0\), this ratio is always contained in the interval \([0, 1)\), regardless of the time step size.
This result establishes a direct structural analogy with the Explicit Upwind Range Condition.
In the explicit case, the update is limited by the inflow from the \textit{old} time level (\(u_{i-1}^n - u_i^n\)).
In the implicit case, the update is limited by the inflow from the \textit{new} time level (\(u_{i-1}^{n+1} - u_i^n\)).
We define \eqref{eq:implicit_upwind_range_condition} as the \textit{Implicit Upwind Range Condition} and will utilize it to construct oscillation-free high-resolution implicit schemes.

\paragraph{Connection to Algebraic Constraints}
It is important to note that this geometric condition is the 1D scalar equivalent of the \textit{M-Matrix property} required for monotonicity in general implicit systems, as analyzed by Kuzmin \cite{kuzmin2014hierarchical}.
In the matrix formulation \(A u^{n+1} = u^n\), monotonicity is guaranteed if \(A\) is an M-matrix (i.e., \(A^{-1} \geq 0\)).
While the M-Matrix property provides a rigorous algebraic guarantee of monotonicity, the geometric perspective offered by the Implicit Upwind Range Condition is more intuitive for constructing and understanding the behavior of slope limiters in a finite volume context.
